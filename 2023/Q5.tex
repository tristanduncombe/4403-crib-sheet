\subsection{Question 5}

Recall the datatype definition of a list and function Length from the lectures.
\begin{verbatim}
datatype List<T> = Nil | Cons(head: T, tail: List<T>)
function Length<T>(xs: List<T>): nat {
        match xs
        case Nil => 0
        case Cons(_ , tail) => 1 + Length(tail)
}    
\end{verbatim}

\subsubsection{(a)}
Write a function Remove which takes a list and an index i of the list as arguments and
returns a new list with the element at index i removed. For example, given the list [0, 1, 2, 3] and
index 2, the function should return [0, 1, 3].

\begin{verbatim}
function Remove<T>(xs: List<T>, i: nat): List<T>
  requires i < Length(xs)
{
  match xs
  case Cons(x, tail) => if i == 0 then tail
        else Cons(x, Remove(tail, i-1))
}
\end{verbatim}

\subsubsection{(b)}
The length of the list returned by Remove is one less than the length of the list provided as
an argument. Show how this would be stated as an intrinsic property of Remove.

The following is added to the function above
\begin{verbatim}
  ensures Length(Remove(xs,i)) == Length(xs) - 1
\end{verbatim}

\subsubsection{(c)}
State the property of part (b) as an extrinsic property of Remove.
\begin{verbatim}
lemma LengthRemove<T>(xs: List<T>, i: nat)
  requires i < Length(xs)
  ensures Length(Remove(xs,i)) == Length(xs) - 1
\end{verbatim}