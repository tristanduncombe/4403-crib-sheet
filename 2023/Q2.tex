\subsubsection{Question 2}
\subsubsubsection{(a)}
Write a specification for a Dafny method to reverse an array. For example, given the array
[1, 2, 3, 4, 5] the method will change it to [5, 4, 3, 2, 1]. Note that the method should modify an
existing array, not create a new one.

\begin{verbatim}
method Reverse(a: array)
  modifies a
  ensures forall i :: 0 <= i < a.Length ==>
                        a[i] == old(a[a.Length-1-i])
\end{verbatim}

\subsubsubsection{(b)}

Based on your specification, provide a loop specification (guard and invariant) for the
Reverse method, and code to initialise the loop variables.

\begin{verbatim}
var n := 0;
while n < a.Length/2
  invariant 0 <= n <= a.Length/2
  invariant forall i :: 0 <= i < n
                ==> a[i] == old(a[a.Length-1-i])
  invariant forall i :: a.Length-n <= i < a.Length
                ==> a[i] == old(a[a.Length-1-i])
  invariant forall i :: n <= i < a.Length-n 
                ==> a[i] == old(a[i])
\end{verbatim}
The second and third invariants are instances of the Replace a Constant by a Variable loop
design technique. In the second invariant, the constant a.Length is replaced by n. In the third
invariant, the constant 0 is replaced by a.Length-n.
The final invariant states that nothing between indices n and a.Length-n have been changed by
the loop. This is similar to the additional invariant we required for the IncrementArray example
in Week 5.

\subsubsubsection{(c)}
Provide a termination metric for the loop.
\begin{verbatim}
decreases a.Length/2 - n
\end{verbatim}