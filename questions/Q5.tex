\section{Question 5}

\subsection{Lemmas}
\textbf{lemma} \textit{name}$(x_1: \textit{T}, x_2: \textit{T}, \ldots, x_n: \textit{T})$ \\
        \tab \textbf{requires} P \\
        \tab \textbf{ensures} R \\
\{ \}

Lemmas can be called in a method to \textbf{prove} the lemmas property from that point onwards.

\subsubsection{Weakest Precondition}
\textbf{wp}(M(E), Q)  =  \verb!P[x\E] && (R[x\E] ==> Q)!

\subsubsection{Calc}
To prove a lemma by hand, you can add a \textbf{calc} section into the lemmas body, where \textit{$\gamma$} is the default transitive operator between lines. \\
$\textbf{calc} \, \textit{$\gamma$} \, \{$ \\
$\tab    5*(x + 3);$ \\
$\tab == 5*x + 5*3;$ \\
$\tab == 5x + 15;$ \\
$\}$

You can use use any transitive operator between lines (e.g. $==>$). If no default operator is specific, the default is $==$.

The \textbf{calc} statements can also be added inline within a method instead of creating and calling a lemma.

\subsubsection{Induction}
Lemmas can also be used to prove using induction by recursively calling the lemma in the body. E.g. \\
\textbf{lemma} SumLemma(a: \textit{array<int>}, i: \textit{int}, j: \textit{int}) \\
        \tab \textbf{requires} P \\
        \tab \textbf{ensures} R \\
\{ \\
        \tab if i == j \{\} // base case: Dafny can prove\\
        \tab else \{\\
        \tab    \tab SumLemma(a, i+1, j); // inductive case\\
        \tab \}\\
\}

\subsection{Functional Programming}
Key features:
\begin{itemize}
        \item Program structures as mathematical functions
        \item Data is immutable (i.e. no heap, no side effects)
\end{itemize}

\subsubsection{Match}

\textbf{Match} is dafny's version of a switch statement, but it must cover all cases.\\
\tab \textbf{match} $x$\\
\tab \textbf{case} $c_1$\\
\tab \textbf{case} $c_2$\\
\tab \ldots\\
\tab \textbf{case} $c_n$\\

\subsubsection{Descriminators}
Discriminators can be used to check if a variable is a given type. E.g. xs.Nil\textbf{?} checks if xs is type Nil.

\subsubsection{Destructors}
Destructors are used to access data in a composite datatype. E.g. for a variable xs of the datatype\\
\textbf{datatype} \verb!List<T>! = Nil | Cons(head: T, tail: \verb!List<T>!),\\
head can be accessed using xs.head. Similarly tail can be accessed using xs.tail.

\subsubsection{Instrinsic vs Extrinsic Property}
\begin{itemize}
        \item An intrinsic property is a property defined within a specification.
        \item An extrinsic property is a property defined externally using a lemma.
        \item Methods in Dafny are opaque, so all properties in the specification are intrinsic.
        \item Functions are transparent, so properties can be intrinsic or extrinsic.
        \item Intrinsic properties are available every time we apply a function, whereas extrinsic properties are only available if we call the lemma.
        \item Having all properties exposed instrinsicly can lead to long verification times, so only define properties intrinsicly if they will be required for all applications of the function.
\end{itemize}