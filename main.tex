\documentclass[a4paper]{article}
\usepackage{multicol}
\usepackage{calc}
\usepackage{ifthen}
\usepackage[landscape]{geometry}
\usepackage{hyperref}
\usepackage{amsmath}
\usepackage{multicol}
\usepackage{fancyvrb}
\usepackage{enumitem}
\usepackage{listings}
\usepackage{calrsfs}
\usepackage{enumitem}% http://ctan.org/pkg/enumitem
\setlist[itemize]{noitemsep, topsep=0pt, leftmargin=*}

\DeclareMathAlphabet{\pazocal}{OMS}{zplm}{m}{n}
\newcommand{\stararrow}{\stackrel{*}{\Rightarrow}}

% To make this come out properly in landscape mode, do one of the following
% 1.
%  pdflatex latexsheet.tex
%
% 2.
%  latex latexsheet.tex
%  dvips -P pdf  -t landscape latexsheet.dvi
%  ps2pdf latexsheet.ps


% If you're reading this, be prepared for confusion.  Making this was
% a learning experience for me, and it shows.  Much of the placement
% was hacked in; if you make it better, let me know...


% 2008-04
% Changed page margin code to use the geometry package. Also added code for
% conditional page margins, depending on paper size. Thanks to Uwe Ziegenhagen
% for the suggestions.

% 2006-08
% Made changes based on suggestions from Gene Cooperman. <gene at ccs.neu.edu>


% To Do:
% \listoffigures \listoftables
% \setcounter{secnumdepth}{0}

\geometry{top=5mm,left=4mm,right=4mm,bottom=5mm}

% Turn off header and footer
\pagestyle{empty}
 

% Redefine section commands to use less space
\makeatletter
\renewcommand{\section}{\@startsection{section}{1}{0mm}%
                                {-1ex plus -.5ex minus -.2ex}%
                                {0.5ex plus .2ex}%
                                {\normalfont\fontsize{12pt}{6pt}\bfseries}}
\renewcommand{\subsection}{\@startsection{subsection}{2}{0mm}%
                                {-1explus -.5ex minus -.2ex}%
                                {0.5ex plus .2ex}%
                                {\normalfont\fontsize{12pt}{6pt}\bfseries\underline}}
\renewcommand{\subsubsection}{\@startsection{subsubsection}{3}{0mm}%
                                {-1ex plus -.5ex minus -.2ex}%
                                {1ex plus .2ex}%
                                {\normalfont\fontsize{10pt}{6pt}\bfseries}}                   
\newcommand{\subsubsubsection}{\@startsection{subsubsection}{4}{0mm}%
                                {-1ex plus -.5ex minus -.2ex}%
                                {1ex plus .2ex}%
                                {\normalfont\fontsize{10pt}{6pt}\bfseries\itshape}}
                                                 
\newcommand{\subsubsubsubsection}{\@startsection{subsubsection}{5}{0mm}%
                                {-1ex plus -.5ex minus -.2ex}%
                                {1ex plus .2ex}%
                                {\normalfont\fontsize{10pt}{6pt}\itshape}}
                                \makeatother

\newcommand{\imp}{\Rightarrow}

% Define BibTeX command
\def\BibTeX{{\rm B\kern-.05em{\sc i\kern-.025em b}\kern-.08em
    T\kern-.1667em\lower.7ex\hbox{E}\kern-.125emX}}

% Don't print section numbers
\setcounter{secnumdepth}{0}


\setlength{\parindent}{0pt}
\setlength{\parskip}{0pt plus 0.2ex}

\newcommand\tab[1][0.1cm]{\hspace*{#1}}

% -----------------------------------------------------------------------

\begin{document}

\raggedright
\fontsize{10pt}{6pt}\selectfont
\begin{multicols}{4}


% multicol parameters
% These lengths are set only within the two main columns
\setlength{\columnseprule}{0.25pt}
\setlength{\premulticols}{1pt}
\setlength{\postmulticols}{1pt}
\setlength{\multicolsep}{0.5pt}
\setlength{\columnsep}{0.5pt}
\setlength{\tabcolsep}{2pt}
\setlength{\abovedisplayskip}{2pt}
\setlength{\belowdisplayskip}{0pt}
\setlength{\topsep}{0pt}
\setlength{\partopsep}{0pt}
\setlength{\baselineskip}{5pt}
\setlength{\fboxsep}{0.5pt}

\subsection{Parsing Theory}

\subsubsection{Context Free Grammars}

Basic Example of Context Free Grammar

$E \rightarrow E \ Op \ E $ \\
$E \rightarrow "(" E ")"$ \\
$E \rightarrow number$ \\
$Op \rightarrow "+"$ \\
$Op \rightarrow "-"$ \\
$Op \rightarrow "*"$ \\
Has start symbol $E$, nonterminals $\{E, Op\}$, and terminals
$\{“(”, “)”, number , “+”, “-”, “*”\}$

A context-free grammar consists of:
\begin{itemize}
\item A finite set, $\sum$,  of terminal symbols.
\item A finite nonempty set of nonterminal symbols (disjoint from the terminal symbols).
\item A finite nonempty set of producions of the form of $A \rightarrow \alpha$, where $A$ is a nonterminal symbol, and $\alpha$ is a possibly empty sequence of symbols, each of which is either a terminal of nonterminal symbol.
\item A start symbol that must be a nonterminal symbol
\end{itemize}

\subsubsubsection{Directly Derives}
If there is a production in the form of $N \rightarrow \gamma$ then we can directly derive $\alpha \ N \beta \ \rightarrow \alpha \ \gamma \ \beta$, where $\alpha$ and $\beta$ are possibly empty sequences of terminal and nonterminal symbols.

\subsubsubsection{Derives}
Given a sequence of terminal and nonterminal symbols, $\alpha$, derives a sequence $\beta$, written $\alpha \stackrel{*}{\Rightarrow} \beta$ if there is a finitie sequence of zero or more direct derivation steps that start from $\alpha$ and finishing with $\beta$, there must be one or more sequence $\gamma_1, \gamma_2, \ldots, \gamma_n$ such that $\alpha = \gamma_0 \Rightarrow \gamma_1 \Rightarrow \ldots \Rightarrow \gamma_n = \beta$. Note that zero steps are allowed.

\subsubsubsection{Nullable}
A possibly empty sequence of symbols, $\alpha$, is nullable if  $\alpha \stackrel{*}{\Rightarrow} \epsilon$ or $\alpha \stackrel{*}{\Rightarrow}$
Nullable rules
\begin{itemize}
        \item $\epsilon is nullable$
        \item any terminal symbol is not nullable
        \item a sequence of symbols is nullable if all of its constructs are nullable
        \item a set of alternatives is nullable if any of its constructs are nullable
        \item EBNF constructs for optionals and repetitions are nullable
        \item a nonterminal is nullable if there is a production with a nullable right-hand side
\end{itemize}

\subsubsubsection{Language}
The formal language $\pazocal{L}(G)$ corresponding to a Grammar, G, is:
$\pazocal{L}(G) = \{t \in seq \sum | S \stararrow t \}$

where S is the start symbol of $G$ and $\sum$ is its set of terminal symbols.

\subsubsubsection{Sentences and Sentenial Form}
A sequence of terminal symbols $t$ such that $S \stararrow t$ is called a \emph{sentence} of the language.

\subsubsubsection{Ambiguous grammars for sequence}
A grammar, \emph{G}, is ambiguous for a sentence, \emph{t}, in $\pazocal{L}$, if there is more than one parse tree for tree for \emph{t}.
\subsubsubsection{Ambiguous grammars}
A grammar, \emph{G}, is ambiguous if it is ambiguous for any setence in $\pazocal{L}(G)$.
\subsubsubsection{Left and right associative operators}
To remove the ambiguit and treat \emph{"-"} as  a left associate operator (as usual) we can rewrite the grammar to

$E \rightarrow E \ "-" \ T$ \\
$E \rightarrow T$ \\
$T \rightarrow N$ \\
and to treat \emph{"-"} as right associative we use
$E \rightarrow T "-" E$ \\
$E \rightarrow T$ \\
$T \rightarrow N$ \\

\subsection{Recusive-Descent Parsing}
\subsubsection{Translating EBNF to BNF}

Replace optional construct $[S]$ by a new nonterminal $OptS$ where $OptS \rightarrow S | \epsilon$.\\
Replace grouping construct ($S$) by a new nonterminal $GrpS$ where $GrpS \rightarrow S$.\\
Replace repetition construct \{$S$\} by a new nonterminal $RepS$ where $RepS \rightarrow S RepF | \epsilon$.\\

\subsection{First \& Follow Sets}

\subsubsection{First Sets}
The first set for a construct $\alpha$ records
\begin{itemize}
    \item the set of terminal symbols $\alpha$ can start with, and
    \item if $\alpha$ is nullable, it contains the empty string $\epsilon$ to indicate that.
\end{itemize}

\subsubsubsection{Calculating First Sets}
Let
\begin{itemize}
    \item $\alpha$ be a terminal symbok,
    \item $\alpha_1,\alpha_2,\dots,\alpha_n$ be strings of (terminal and nonterminal) symbols, and
    \item $A$ be a nonterminal symbol defined by a single production
\end{itemize}
\begin{align*}
    A \rightarrow \alpha_1 | \alpha_2 | \dots | \alpha_n,
\end{align*}
then
\begin{align*}
    Fst(\epsilon) &= \{\epsilon\}\\
    Fst(\alpha) &= \{\alpha\}\\
    Fst(\alpha_1 | \alpha_2 | \dots | \alpha_n) &= Fst(\alpha_1) \cup Fst(\alpha_2)\\
                                                  &\ \ \ \ \cup \dots \cup Fst(\alpha_n)\\  
    Fst(A) &= Fst(\alpha_1 | \alpha_2 | \dots | \alpha_n)\\
\end{align*}

Let $S_1,S_2,\dots,S_n$ be (terminal or nonterminal) symbols, then
\begin{align*}
    F&st(S_1 S_2 \dots S_n) =\\
    &Fst(S_1) - \{\epsilon\}\\
    &\cup Fst(S_2) - \{\epsilon\} &&\text{if $S_1$ is nullable}\\
    &\cup \dots\\
    &\cup Fst(S_i) - \{\epsilon\} &&\text{if $S_1$-$S_{i-1}$ is nullable}\\
    &\cup \dots\\
    &\cup Fst(S_n) - \{\epsilon\} &&\text{if $S_1$-$S_{n-1}$ is nullable}\\
    &\cup \{\epsilon\} &&\text{if $S_1$-$S_n$ is nullbale}
\end{align*}

\subsubsubsection{Calculating Algorithmically}
Start with the first sets for all nonterminals being the empty set and note that the fisrt set for every terminal symbol, $\alpha$, is the singleton set $\{\alpha\}$.
We then make a pass over all production in a grammer considering all atlternatives and process as follows.
\begin{itemize}
    \item If there is a production of the form $N \rightarrow \epsilon$, we add $\epsilon$ to the first set of $N$.
    \item If there is a production of the form $N \rightarrow S_1 S_2 \dots S_n$, then for each $i \in 1..n$, if for all $j \in 1..i-1$, $S_j$ is nullabe, we add the current first set for $S_i$ minus $\epsilon$ to the first set for $N$.
    \item If every construct $S_1,...,S_n$ is nullabe, we add $\epsilon$ to the first set for $N$. 
\end{itemize}
We repeat the the passes until no set is modified in the pass, in which case we are finished.
\subsubsubsubsection{Example}
\begin{align*}
    A &\rightarrow B\,x | C\\
    B &\rightarrow C\,y | D\\
    C &\rightarrow D\,z | \epsilon\\
    D &\rightarrow A\,w
\end{align*}
\begin{center}
\begin{tabular}{|c|c|c|c|c|c|}
    \hline
    $A$ & $\{\}$ & $\{\epsilon\}$ & $\{\epsilon,y\}$ &  $\{\epsilon,y,w\}$ &  $\{\epsilon,y,w\}$\\
    \hline
    $B$ & $\{\}$ & $\{y\}$ & $\{y\}$ &  $\{y,w\}$ &  $\{y,w\}$\\
    \hline
    $C$ & $\{\epsilon\}$ & $\{\epsilon\}$ & $\{\epsilon,w\}$ &  $\{\epsilon,w,y\}$ &  $\{\epsilon,w,y\}$\\
    \hline
    $D$ & $\{\}$ & $\{w\}$ & $\{w,y\}$ &  $\{w,y\}$ &  $\{w,y\}$\\
    \hline
\end{tabular}
\end{center}



\subsubsection{Follow Sets}
The follow set for a nonterminal, $N$, is the set of terminal symbols that may follow $N$ in any context within the grammar.
End-of-file is represented by the special terminal symbol $\$$ in which $a$ follows $N$.

A nonterminal, $N$, is followed by a terminal symbol, $a$, if there is a derivation from $S\$$

\subsubsubsection{Calculating Follow Sets}
We compute the Follow set for a nonterminal, $N$, using two rules.
\begin{itemize}
    \item If there is a production of the form
    \begin{align*}
        A \rightarrow \alpha N \beta
    \end{align*}
    then any symbols that can start $\beta$ can follow $N$, and hence $Follow(N)$ must include all the terminal symbols in $First(\beta)$. Note that $\epsilon$ is not included even if it appears in $First(\beta)$.
    \begin{align*}
        First(\beta) - \{\epsilon\} \subseteq Follow(N)
    \end{align*}
    \item If there is a production of the form
    \begin{align*}
        A \rightarrow \alpha N \beta
    \end{align*}
    and $\beta$ is nullable, then any token that can follow $A$ can also follow $N$. Hence,
    \begin{align*}
        Follow(A) \subseteq Follow(N)
    \end{align*}
    The case where $\beta$ is nullable includes the case when $\beta$ is empty and the production is of the form
    \begin{align*}
        A \rightarrow \alpha N
    \end{align*}
\end{itemize}

\subsubsubsection{Calculating Algorithmically}
Start with all nonterminal symbols having an empty follow set, $\{\}$, except for the start symbol, $S$, which has the follow set $\{\$\}$.

We make a pass through the grammar examining the right side of every production. 
For each occurence of a nonterminal within the right side of some production, 
we augment the Follow set for that nonterminal according to the following process. 
Assume we are processing an occurence of $N$ and the production is of the form
\begin{align*}
    A \rightarrow \alpha N \beta
\end{align*}
we add $First(\beta) - \{\epsilon\}$ to the Follow set computer for $N$ so far, 
and if $\beta$ is nullable, we also add the current Follow set for $A$ to the Follow set for $N$.

After making a complete pass, we repeat the process with the Follow sets computed so far until no Follow sets are modified, in which case we are done.
\subsubsubsubsection{Example}
\begin{equation*}
    \begin{aligned}[c]
        S &\rightarrow x A B\\
        A &\rightarrow y | z B\\
        B &\rightarrow \epsilon | A x
    \end{aligned}
    \qquad 
    \begin{aligned}[c]
        Fst(S) &= \{x\}\\
        Fst(A) &= \{y, z\}\\
        Fst(B) &= \{\epsilon, y, z\}
    \end{aligned}
\end{equation*}

\begin{center}
    \begin{tabular}{|c|c|c|c|c|}
        \hline
        $S$ & $\{\$\}$ & $\{\$\}$ & $\{\$\}$ & $\{\$\}$\\
        \hline
        $A$ & $\{\}$ & $\{y,z,\$,x\}$ & $\{y,z,\$,x\}$ & $\{y,z,\$,x\}$\\
        \hline
        $B$ & $\{\}$ & $\{\$,y,z\}$ & $\{\$,y,z,x\}$ & $\{\$,y,z,x\}$\\
        \hline
    \end{tabular}
\end{center}

\subsubsection{LL(1) Grammar}
A BNF grammer is LL(1) if for each nonterminal, $N$, wher $N \rightarrow \alpha_1 | \alpha_2 | \dots | \alpha_n$,
\begin{itemize}
    \item the First sets for each pair of alternatives for $N$ are disjoint, and
    \item if $N$ is nullable, $First(N)$ and $Follow(N)$ are disjoin.
\end{itemize}


\subsection{Left Factoring \& Left Recursion}

\subsubsection{Left Factoring Productions}
Not all EBNF grammars are suitable for Recursive-Descent Parsing, 
however, sometimes we can rewrite them into a form that is suitable.

\subsubsubsection{Left Factor Rewriting Rule}
To remove the left factor from
\begin{align*}
    A \rightarrow \alpha\ \beta\ |\ \alpha\ \gamma,
\end{align*}
we can rewrite the production using an aditional nonterminal $A'$ as
\begin{align*}
    A &\rightarrow \alpha\ A'\\
    A' &\rightarrow \beta\ |\ \gamma
\end{align*}
\subsubsection{Left Recursive Productions}
A production of the form
\begin{align*}
    E \rightarrow E\ +\ T\ |\ T
\end{align*}
is not suitable for RDP because the left recursion in the grammar leads to an infinite recursion.

\subsubsubsection{Immediate Left Recursion Rewriting Rule (Simple)}
To remove the left recursion from
\begin{align*}
    A \rightarrow A\ \alpha\ |\ \beta,
\end{align*}
we can rewrite the production as
\begin{align*}
    A &\rightarrow \beta\ A'\\
    A' &\rightarrow \epsilon\ |\ \alpha\ A'
\end{align*}

\subsubsubsection{Immediate Left Recursion Rewriting Rule (General)}
To remove the left recursion from the general case
\begin{align*}
    A \rightarrow A\alpha_1|A\alpha_2|\dots|A\alpha_n|\beta_1|\beta_2|\dots|\beta_m,
\end{align*}
we can use grouping to see the structure in the same form as the simple case
\begin{align*}
    A \rightarrow A(\alpha_1|\alpha_2|\dots|\alpha_n)|(\beta_1|\beta_2|\dots|\beta_m)
\end{align*}
which allows us to rewrite the production as follows,
\begin{align*}
    A &\rightarrow (\beta_1|\beta_2|\dots|\beta_m)A'\\
    A' &\rightarrow \epsilon|(\alpha_1|\alpha_2|\dots|\alpha_n)A'
\end{align*}

\subsubsubsection{Indirect Left Recursion Rewriting Rule}
The following productions have indirect recursion from $A \rightarrow B \rightarrow C \rightarrow A$.
\begin{align*}
    A &\rightarrow B\ \alpha\\
    B &\rightarrow C\ \beta\\
    C &\rightarrow A\ \gamma_1\ |\  \gamma_2
\end{align*}
To remove the recursion we can first collapse $B$ into $A$ as follows.
\begin{align*}
    A &\rightarrow C\ \beta\ \alpha\\
    C &\rightarrow A\ \gamma_1\ |\ \gamma_2
\end{align*}
Then we can collapse $C$ into $A$.
\begin{align*}
    A &\rightarrow (A\ \gamma_1\ |\  \gamma_2)\ \beta\ \alpha\\
    &\ \ \ \ \ \ \ \ \ \ \downarrow\\
    A &\rightarrow A\ \gamma_1\ \beta\ \alpha\ |\ \gamma_2\ \beta\ \alpha
\end{align*}
This now leaves a simple direct left recursion which we can remove as follows.
\begin{align*}
    A &\rightarrow \gamma_2\ \beta\ \alpha\ A'\\
    A' &\rightarrow \gamma_1\ \beta\ \alpha\ A'
\end{align*}

\scriptsize

\end{multicols}
\end{document}
