\documentclass[a4paper]{article}
\usepackage{multicol}
\usepackage{calc}
\usepackage{ifthen}
\usepackage[landscape]{geometry}
\usepackage{hyperref}
\usepackage{amsmath}
\usepackage{multicol}
\usepackage{fancyvrb}
\usepackage{enumitem}
\usepackage{listings}
\usepackage{calrsfs}

\DeclareMathAlphabet{\pazocal}{OMS}{zplm}{m}{n}
\newcommand{\stararrow}{\stackrel{*}{\Rightarrow}}

% To make this come out properly in landscape mode, do one of the following
% 1.
%  pdflatex latexsheet.tex
%
% 2.
%  latex latexsheet.tex
%  dvips -P pdf  -t landscape latexsheet.dvi
%  ps2pdf latexsheet.ps


% If you're reading this, be prepared for confusion.  Making this was
% a learning experience for me, and it shows.  Much of the placement
% was hacked in; if you make it better, let me know...


% 2008-04
% Changed page margin code to use the geometry package. Also added code for
% conditional page margins, depending on paper size. Thanks to Uwe Ziegenhagen
% for the suggestions.

% 2006-08
% Made changes based on suggestions from Gene Cooperman. <gene at ccs.neu.edu>


% To Do:
% \listoffigures \listoftables
% \setcounter{secnumdepth}{0}


% This sets page margins to .5 inch if using letter paper, and to 1cm
% if using A4 paper. (This probably isn't strictly necessary.)
% % If using another size paper, use default 1cm margins.
\ifthenelse{\lengthtest { \paperwidth = 11in}}
	{ \geometry{top=.20in,left=.20in,right=.20in,bottom=.20in} }
	{\ifthenelse{ \lengthtest{ \paperwidth = 297mm}}
		{\geometry{top=1cm,left=1cm,right=1cm,bottom=1cm} }
		{\geometry{top=1cm,left=1cm,right=1cm,bottom=1cm} }
	}

% Turn off header and footer
\pagestyle{empty}
 

% Redefine section commands to use less space
\makeatletter
\renewcommand{\section}{\@startsection{section}{1}{0mm}%
                                {-1ex plus -.5ex minus -.2ex}%
                                {0.5ex plus .2ex}%
                                {\normalfont\normalsize\bfseries}}
\renewcommand{\subsection}{\@startsection{subsection}{2}{0mm}%
                                {-1explus -.5ex minus -.2ex}%
                                {0.5ex plus .2ex}%
                                {\normalfont\small\bfseries}}
\renewcommand{\subsubsection}{\@startsection{subsubsection}{3}{0mm}%
                                {-1ex plus -.5ex minus -.2ex}%
                                {1ex plus .2ex}%
                                {\normalfont\footnotesize\bfseries}}                   
\newcommand{\subsubsubsection}{\@startsection{subsubsection}{4}{0mm}%
                                {-1ex plus -.5ex minus -.2ex}%
                                {1ex plus .2ex}%
                                {\normalfont\scriptsize\bfseries}}
\makeatother
\newcommand{\imp}{\Rightarrow}

% Define BibTeX command
\def\BibTeX{{\rm B\kern-.05em{\sc i\kern-.025em b}\kern-.08em
    T\kern-.1667em\lower.7ex\hbox{E}\kern-.125emX}}

% Don't print section numbers
\setcounter{secnumdepth}{0}


\setlength{\parindent}{0pt}
\setlength{\parskip}{0pt plus 0.2ex}

\newcommand\tab[1][0.1cm]{\hspace*{#1}}

% -----------------------------------------------------------------------

\begin{document}

\raggedright
\fontsize{10pt}{6pt}\selectfont
\begin{multicols}{4}


% multicol parameters
% These lengths are set only within the two main columns
%\setlength{\columnseprule}{0.25pt}
\setlength{\premulticols}{1pt}
\setlength{\postmulticols}{1pt}
\setlength{\multicolsep}{1pt}
\setlength{\columnsep}{2pt}

\setlength{\topsep}{1pt}\setlength{\partopsep}{2pt}

\begin{center}
     \Large{\textbf{COMP4403 Crib Sheet}} \\
\end{center}

\subsection{Parsing Theory}

\subsubsection{Context Free Grammars}

Basic Example of Context Free Grammar

$E \rightarrow E \ Op \ E $ \\
$E \rightarrow "(" E ")"$ \\
$E \rightarrow number$ \\
$Op \rightarrow "+"$ \\
$Op \rightarrow "-"$ \\
$Op \rightarrow "*"$ \\
Has start symbol $E$, nonterminals $\{E, Op\}$, and terminals
$\{“(”, “)”, number , “+”, “-”, “*”\}$

A context-free grammar consists of:
\begin{itemize}
\item A finite set, $\sum$,  of terminal symbols.
\item A finite nonempty set of nonterminal symbols (disjoint from the terminal symbols).
\item A finite nonempty set of producions of the form of $A \rightarrow \alpha$, where $A$ is a nonterminal symbol, and $\alpha$ is a possibly empty sequence of symbols, each of which is either a terminal of nonterminal symbol.
\item A start symbol that must be a nonterminal symbol
\end{itemize}

\subsubsubsection{Directly Derives}
If there is a production in the form of $N \rightarrow \gamma$ then we can directly derive $\alpha \ N \beta \ \rightarrow \alpha \ \gamma \ \beta$, where $\alpha$ and $\beta$ are possibly empty sequences of terminal and nonterminal symbols.

\subsubsubsection{Derives}
Given a sequence of terminal and nonterminal symbols, $\alpha$, derives a sequence $\beta$, written $\alpha \stackrel{*}{\Rightarrow} \beta$ if there is a finitie sequence of zero or more direct derivation steps that start from $\alpha$ and finishing with $\beta$, there must be one or more sequence $\gamma_1, \gamma_2, \ldots, \gamma_n$ such that $\alpha = \gamma_0 \Rightarrow \gamma_1 \Rightarrow \ldots \Rightarrow \gamma_n = \beta$. Note that zero steps are allowed.

\subsubsubsection{Nullable}
A possibly empty sequence of symbols, $\alpha$, is nullable if  $\alpha \stackrel{*}{\Rightarrow} \epsilon$ or $\alpha \stackrel{*}{\Rightarrow}$
Nullable rules
\begin{itemize}
        \item $\epsilon is nullable$
        \item any terminal symbol is not nullable
        \item a sequence of symbols is nullable if all of its constructs are nullable
        \item a set of alternatives is nullable if any of its constructs are nullable
        \item EBNF constructs for optionals and repetitions are nullable
        \item a nonterminal is nullable if there is a production with a nullable right-hand side
\end{itemize}

\subsubsubsection{Language}
The formal language $\pazocal{L}(G)$ corresponding to a Grammar, G, is:
$\pazocal{L}(G) = \{t \in seq \sum | S \stararrow t \}$

where S is the start symbol of $G$ and $\sum$ is its set of terminal symbols.

\subsubsubsection{Sentences and Sentenial Form}
A sequence of terminal symbols $t$ such that $S \stararrow t$ is called a \emph{sentence} of the language.

\subsubsubsection{Ambiguous grammars for sequence}
A grammar, \emph{G}, is ambiguous for a sentence, \emph{t}, in $\pazocal{L}$, if there is more than one parse tree for tree for \emph{t}.
\subsubsubsection{Ambiguous grammars}
A grammar, \emph{G}, is ambiguous if it is ambiguous for any setence in $\pazocal{L}(G)$.
\subsubsubsection{Left and right associative operators}
To remove the ambiguit and treat \emph{"-"} as  a left associate operator (as usual) we can rewrite the grammar to

$E \rightarrow E \ "-" \ T$ \\
$E \rightarrow T$ \\
$T \rightarrow N$ \\
and to treat \emph{"-"} as right associative we use
$E \rightarrow T "-" E$ \\
$E \rightarrow T$ \\
$T \rightarrow N$ \\

\subsection{Recusive-Descent Parsing}
\subsubsection{Translating EBNF to BNF}

Replace optional construct $[S]$ by a new nonterminal $OptS$ where $OptS \rightarrow S | \epsilon$.\\
Replace grouping construct ($S$) by a new nonterminal $GrpS$ where $GrpS \rightarrow S$.\\
Replace repetition construct \{$S$\} by a new nonterminal $RepS$ where $RepS \rightarrow S RepF | \epsilon$.\\

\scriptsize

\end{multicols}
\end{document}
