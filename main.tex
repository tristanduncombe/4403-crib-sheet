\documentclass[a4paper]{article}
\usepackage{multicol}
\usepackage{calc}
\usepackage{ifthen}
\usepackage[landscape]{geometry}
\usepackage{hyperref}
\usepackage{amsmath}
\usepackage{multicol}
\usepackage{fancyvrb}
\usepackage{enumitem}
\usepackage{listings}
\usepackage{calrsfs}
\usepackage{enumitem}% http://ctan.org/pkg/enumitem
\usepackage{graphicx}
\usepackage{ amssymb }

\graphicspath{ {./images/} }
\setlist[itemize]{noitemsep, topsep=0pt, leftmargin=*}

\DeclareMathAlphabet{\pazocal}{OMS}{zplm}{m}{n}
\newcommand{\stararrow}{\stackrel{*}{\Rightarrow}}

% To make this come out properly in landscape mode, do one of the following
% 1.
%  pdflatex latexsheet.tex
%
% 2.
%  latex latexsheet.tex
%  dvips -P pdf  -t landscape latexsheet.dvi
%  ps2pdf latexsheet.ps


% If you're reading this, be prepared for confusion.  Making this was
% a learning experience for me, and it shows.  Much of the placement
% was hacked in; if you make it better, let me know...


% 2008-04
% Changed page margin code to use the geometry package. Also added code for
% conditional page margins, depending on paper size. Thanks to Uwe Ziegenhagen
% for the suggestions.

% 2006-08
% Made changes based on suggestions from Gene Cooperman. <gene at ccs.neu.edu>


% To Do:
% \listoffigures \listoftables
% \setcounter{secnumdepth}{0}

\geometry{top=5mm,left=4mm,right=4mm,bottom=5mm}

% Turn off header and footer
\pagestyle{empty}
 

% Redefine section commands to use less space
\makeatletter
\renewcommand{\section}{\@startsection{section}{1}{0mm}%
                                {-1ex plus -.5ex minus -.2ex}%
                                {0.5ex plus .2ex}%
                                {\normalfont\fontsize{12pt}{6pt}\bfseries}}
\renewcommand{\subsection}{\@startsection{subsection}{2}{0mm}%
                                {-1explus -.5ex minus -.2ex}%
                                {0.5ex plus .2ex}%
                                {\normalfont\fontsize{12pt}{6pt}\bfseries\underline}}
\renewcommand{\subsubsection}{\@startsection{subsubsection}{3}{0mm}%
                                {-1ex plus -.5ex minus -.2ex}%
                                {1ex plus .2ex}%
                                {\normalfont\fontsize{10pt}{6pt}\bfseries}}                   
\newcommand{\subsubsubsection}{\@startsection{subsubsection}{4}{0mm}%
                                {-1ex plus -.5ex minus -.2ex}%
                                {1ex plus .2ex}%
                                {\normalfont\fontsize{10pt}{6pt}\bfseries\itshape}}
                                                 
\newcommand{\subsubsubsubsection}{\@startsection{subsubsection}{5}{0mm}%
                                {-1ex plus -.5ex minus -.2ex}%
                                {1ex plus .2ex}%
                                {\normalfont\fontsize{10pt}{6pt}\itshape}}
                                \makeatother

\newcommand{\imp}{\Rightarrow}

\newcommand{\image}[2][\columnwidth]{%
    \includegraphics[width=#1]{#2}%
}

% Define BibTeX command
\def\BibTeX{{\rm B\kern-.05em{\sc i\kern-.025em b}\kern-.08em
    T\kern-.1667em\lower.7ex\hbox{E}\kern-.125emX}}

% Don't print section numbers
\setcounter{secnumdepth}{0}


\setlength{\parindent}{0pt}
\setlength{\parskip}{0pt plus 0.2ex}

\newcommand\tab[1][0.1cm]{\hspace*{#1}}

% -----------------------------------------------------------------------

\begin{document}

\raggedright
\fontsize{10pt}{6pt}\selectfont
\begin{multicols}{4}


% multicol parameters
% These lengths are set only within the two main columns
\setlength{\columnseprule}{0.25pt}
\setlength{\premulticols}{1pt}
\setlength{\postmulticols}{1pt}
\setlength{\multicolsep}{0.5pt}
\setlength{\columnsep}{0.5pt}
\setlength{\tabcolsep}{2pt}
\setlength{\abovedisplayskip}{2pt}
\setlength{\belowdisplayskip}{0pt}
\setlength{\topsep}{0pt}
\setlength{\partopsep}{0pt}
\setlength{\baselineskip}{5pt}
\setlength{\fboxsep}{0.5pt}

\subsection{Parsing Theory}

\subsubsection{Context Free Grammars}

Basic Example of Context Free Grammar

$E \rightarrow E \ Op \ E $ \\
$E \rightarrow "(" E ")"$ \\
$E \rightarrow number$ \\
$Op \rightarrow "+"$ \\
$Op \rightarrow "-"$ \\
$Op \rightarrow "*"$ \\
Has start symbol $E$, nonterminals $\{E, Op\}$, and terminals
$\{“(”, “)”, number , “+”, “-”, “*”\}$

A context-free grammar consists of:
\begin{itemize}
\item A finite set, $\sum$,  of terminal symbols.
\item A finite nonempty set of nonterminal symbols (disjoint from the terminal symbols).
\item A finite nonempty set of producions of the form of $A \rightarrow \alpha$, where $A$ is a nonterminal symbol, and $\alpha$ is a possibly empty sequence of symbols, each of which is either a terminal of nonterminal symbol.
\item A start symbol that must be a nonterminal symbol
\end{itemize}

\subsubsubsection{Directly Derives}
If there is a production in the form of $N \rightarrow \gamma$ then we can directly derive $\alpha \ N \beta \ \rightarrow \alpha \ \gamma \ \beta$, where $\alpha$ and $\beta$ are possibly empty sequences of terminal and nonterminal symbols.

\subsubsubsection{Derives}
Given a sequence of terminal and nonterminal symbols, $\alpha$, derives a sequence $\beta$, written $\alpha \stackrel{*}{\Rightarrow} \beta$ if there is a finitie sequence of zero or more direct derivation steps that start from $\alpha$ and finishing with $\beta$, there must be one or more sequence $\gamma_1, \gamma_2, \ldots, \gamma_n$ such that $\alpha = \gamma_0 \Rightarrow \gamma_1 \Rightarrow \ldots \Rightarrow \gamma_n = \beta$. Note that zero steps are allowed.

\subsubsubsection{Nullable}
A possibly empty sequence of symbols, $\alpha$, is nullable if  $\alpha \stackrel{*}{\Rightarrow} \epsilon$ or $\alpha \stackrel{*}{\Rightarrow}$
Nullable rules
\begin{itemize}
        \item $\epsilon is nullable$
        \item any terminal symbol is not nullable
        \item a sequence of symbols is nullable if all of its constructs are nullable
        \item a set of alternatives is nullable if any of its constructs are nullable
        \item EBNF constructs for optionals and repetitions are nullable
        \item a nonterminal is nullable if there is a production with a nullable right-hand side
\end{itemize}

\subsubsubsection{Language}
The formal language $\pazocal{L}(G)$ corresponding to a Grammar, G, is:
$\pazocal{L}(G) = \{t \in seq \sum | S \stararrow t \}$

where S is the start symbol of $G$ and $\sum$ is its set of terminal symbols.

\subsubsubsection{Sentences and Sentenial Form}
A sequence of terminal symbols $t$ such that $S \stararrow t$ is called a \emph{sentence} of the language.

\subsubsubsection{Ambiguous grammars for sequence}
A grammar, \emph{G}, is ambiguous for a sentence, \emph{t}, in $\pazocal{L}$, if there is more than one parse tree for tree for \emph{t}.
\subsubsubsection{Ambiguous grammars}
A grammar, \emph{G}, is ambiguous if it is ambiguous for any setence in $\pazocal{L}(G)$.
\subsubsubsection{Left and right associative operators}
To remove the ambiguit and treat \emph{"-"} as  a left associate operator (as usual) we can rewrite the grammar to

$E \rightarrow E \ "-" \ T$ \\
$E \rightarrow T$ \\
$T \rightarrow N$ \\
and to treat \emph{"-"} as right associative we use
$E \rightarrow T "-" E$ \\
$E \rightarrow T$ \\
$T \rightarrow N$ \\

\subsection{Recusive-Descent Parsing}
\subsubsection{Translating EBNF to BNF}

Replace optional construct $[S]$ by a new nonterminal $OptS$ where $OptS \rightarrow S | \epsilon$.\\
Replace grouping construct ($S$) by a new nonterminal $GrpS$ where $GrpS \rightarrow S$.\\
Replace repetition construct \{$S$\} by a new nonterminal $RepS$ where $RepS \rightarrow S RepF | \epsilon$.\\

\subsection{First \& Follow Sets}

\subsubsection{First Sets}
The first set for a construct $\alpha$ records
\begin{itemize}
    \item the set of terminal symbols $\alpha$ can start with, and
    \item if $\alpha$ is nullable, it contains the empty string $\epsilon$ to indicate that.
\end{itemize}

\subsubsubsection{Calculating First Sets}
Let
\begin{itemize}
    \item $\alpha$ be a terminal symbok,
    \item $\alpha_1,\alpha_2,\dots,\alpha_n$ be strings of (terminal and nonterminal) symbols, and
    \item $A$ be a nonterminal symbol defined by a single production
\end{itemize}
\begin{align*}
    A \rightarrow \alpha_1 | \alpha_2 | \dots | \alpha_n,
\end{align*}
then
\begin{align*}
    Fst(\epsilon) &= \{\epsilon\}\\
    Fst(\alpha) &= \{\alpha\}\\
    Fst(\alpha_1 | \alpha_2 | \dots | \alpha_n) &= Fst(\alpha_1) \cup Fst(\alpha_2)\\
                                                  &\ \ \ \ \cup \dots \cup Fst(\alpha_n)\\  
    Fst(A) &= Fst(\alpha_1 | \alpha_2 | \dots | \alpha_n)\\
\end{align*}

Let $S_1,S_2,\dots,S_n$ be (terminal or nonterminal) symbols, then
\begin{align*}
    F&st(S_1 S_2 \dots S_n) =\\
    &Fst(S_1) - \{\epsilon\}\\
    &\cup Fst(S_2) - \{\epsilon\} &&\text{if $S_1$ is nullable}\\
    &\cup \dots\\
    &\cup Fst(S_i) - \{\epsilon\} &&\text{if $S_1$-$S_{i-1}$ is nullable}\\
    &\cup \dots\\
    &\cup Fst(S_n) - \{\epsilon\} &&\text{if $S_1$-$S_{n-1}$ is nullable}\\
    &\cup \{\epsilon\} &&\text{if $S_1$-$S_n$ is nullbale}
\end{align*}

\subsubsubsection{Calculating Algorithmically}
Start with the first sets for all nonterminals being the empty set and note that the fisrt set for every terminal symbol, $\alpha$, is the singleton set $\{\alpha\}$.
We then make a pass over all production in a grammer considering all atlternatives and process as follows.
\begin{itemize}
    \item If there is a production of the form $N \rightarrow \epsilon$, we add $\epsilon$ to the first set of $N$.
    \item If there is a production of the form $N \rightarrow S_1 S_2 \dots S_n$, then for each $i \in 1..n$, if for all $j \in 1..i-1$, $S_j$ is nullabe, we add the current first set for $S_i$ minus $\epsilon$ to the first set for $N$.
    \item If every construct $S_1,...,S_n$ is nullabe, we add $\epsilon$ to the first set for $N$. 
\end{itemize}
We repeat the the passes until no set is modified in the pass, in which case we are finished.
\subsubsubsubsection{Example}
\begin{align*}
    A &\rightarrow B\,x | C\\
    B &\rightarrow C\,y | D\\
    C &\rightarrow D\,z | \epsilon\\
    D &\rightarrow A\,w
\end{align*}
\begin{center}
\begin{tabular}{|c|c|c|c|c|c|}
    \hline
    $A$ & $\{\}$ & $\{\epsilon\}$ & $\{\epsilon,y\}$ &  $\{\epsilon,y,w\}$ &  $\{\epsilon,y,w\}$\\
    \hline
    $B$ & $\{\}$ & $\{y\}$ & $\{y\}$ &  $\{y,w\}$ &  $\{y,w\}$\\
    \hline
    $C$ & $\{\epsilon\}$ & $\{\epsilon\}$ & $\{\epsilon,w\}$ &  $\{\epsilon,w,y\}$ &  $\{\epsilon,w,y\}$\\
    \hline
    $D$ & $\{\}$ & $\{w\}$ & $\{w,y\}$ &  $\{w,y\}$ &  $\{w,y\}$\\
    \hline
\end{tabular}
\end{center}



\subsubsection{Follow Sets}
The follow set for a nonterminal, $N$, is the set of terminal symbols that may follow $N$ in any context within the grammar.
End-of-file is represented by the special terminal symbol $\$$ in which $a$ follows $N$.

A nonterminal, $N$, is followed by a terminal symbol, $a$, if there is a derivation from $S\$$

\subsubsubsection{Calculating Follow Sets}
We compute the Follow set for a nonterminal, $N$, using two rules.
\begin{itemize}
    \item If there is a production of the form
    \begin{align*}
        A \rightarrow \alpha N \beta
    \end{align*}
    then any symbols that can start $\beta$ can follow $N$, and hence $Follow(N)$ must include all the terminal symbols in $First(\beta)$. Note that $\epsilon$ is not included even if it appears in $First(\beta)$.
    \begin{align*}
        First(\beta) - \{\epsilon\} \subseteq Follow(N)
    \end{align*}
    \item If there is a production of the form
    \begin{align*}
        A \rightarrow \alpha N \beta
    \end{align*}
    and $\beta$ is nullable, then any token that can follow $A$ can also follow $N$. Hence,
    \begin{align*}
        Follow(A) \subseteq Follow(N)
    \end{align*}
    The case where $\beta$ is nullable includes the case when $\beta$ is empty and the production is of the form
    \begin{align*}
        A \rightarrow \alpha N
    \end{align*}
\end{itemize}

\subsubsubsection{Calculating Algorithmically}
Start with all nonterminal symbols having an empty follow set, $\{\}$, except for the start symbol, $S$, which has the follow set $\{\$\}$.

We make a pass through the grammar examining the right side of every production. 
For each occurence of a nonterminal within the right side of some production, 
we augment the Follow set for that nonterminal according to the following process. 
Assume we are processing an occurence of $N$ and the production is of the form
\begin{align*}
    A \rightarrow \alpha N \beta
\end{align*}
we add $First(\beta) - \{\epsilon\}$ to the Follow set computer for $N$ so far, 
and if $\beta$ is nullable, we also add the current Follow set for $A$ to the Follow set for $N$.

After making a complete pass, we repeat the process with the Follow sets computed so far until no Follow sets are modified, in which case we are done.
\subsubsubsubsection{Example}
\begin{equation*}
    \begin{aligned}[c]
        S &\rightarrow x A B\\
        A &\rightarrow y | z B\\
        B &\rightarrow \epsilon | A x
    \end{aligned}
    \qquad 
    \begin{aligned}[c]
        Fst(S) &= \{x\}\\
        Fst(A) &= \{y, z\}\\
        Fst(B) &= \{\epsilon, y, z\}
    \end{aligned}
\end{equation*}

\begin{center}
    \begin{tabular}{|c|c|c|c|c|}
        \hline
        $S$ & $\{\$\}$ & $\{\$\}$ & $\{\$\}$ & $\{\$\}$\\
        \hline
        $A$ & $\{\}$ & $\{y,z,\$,x\}$ & $\{y,z,\$,x\}$ & $\{y,z,\$,x\}$\\
        \hline
        $B$ & $\{\}$ & $\{\$,y,z\}$ & $\{\$,y,z,x\}$ & $\{\$,y,z,x\}$\\
        \hline
    \end{tabular}
\end{center}

\subsubsection{LL(1) Grammar}
A BNF grammer is LL(1) if for each nonterminal, $N$, wher $N \rightarrow \alpha_1 | \alpha_2 | \dots | \alpha_n$,
\begin{itemize}
    \item the First sets for each pair of alternatives for $N$ are disjoint, and
    \item if $N$ is nullable, $First(N)$ and $Follow(N)$ are disjoin.
\end{itemize}


\subsection{Bottom Up Parsing}

\subsubsection{Shift/Reduce Parsing}
Makes use of a parse stack and has three actions:
\begin{itemize}
    \item \textbf{shift} - push the next input symbol onto the stack,
    \item \textbf{reduce} - if a sequence of symbols on the top of the stack, $\alpha$, 
    matches the right side of some production $N \rightarrow \alpha$, then the sequence 
    $\alpha$ on top of the stack is replaced by $N$.
    \item \textbf{accept} - if the stack contains just the start symbol and there is no
    input left, the input has been recognised and is accepted.
\end{itemize}

\subsubsubsection{LR(0) Grammars}
An LR(0) parsing item is of the form
\begin{align*}
    N \rightarrow \alpha \bullet \beta
\end{align*}
which indicates that, in matching $N$, $\alpha$ has been matched and $\beta$ is yet to 
be matched where
\begin{itemize}
    \item $N$ is a nonterminal symbol,
    \item $\alpha$ and $\beta$ are possibly empty sequences of (terminal and nonterminal) 
    symbols such that $N \rightarrow \alpha\ \beta$ is a production of the grammer, and
    \item $\bullet$ marks the current position in matching the right side.
\end{itemize}

\subsubsubsubsection{Parsing Automaton}
An LR(0) parsing automaton consists of
\begin{itemize}
    \item a finite set of states, each of which consists of a set of LR(0) parsing items, and
    \item transitions between states, each of which is labelled by a transition symbol.
\end{itemize}
Each state in an LR(0) parsing automaton must have only one associated parsing action.

\subsubsubsubsection{Automaton States}
If a state has an LR(0) item of the form
\begin{align*}
    N &\rightarrow \alpha \bullet M \beta \\
\end{align*}
where the nonterminal $M$ has productions
\begin{align*}
    M &\rightarrow \alpha_1 \\
    M &\rightarrow \alpha_2 \\
    &\vdots \\
    M &\rightarrow \alpha_m \\
\end{align*}
then the state also includes the derived items
\begin{align*}
    M &\rightarrow \bullet \alpha_1 \\
    M &\rightarrow \bullet \alpha_2 \\
    &\vdots \\
    M &\rightarrow \bullet \alpha_m
\end{align*}

\subsubsubsubsection{Goto States}
If a state $s_i$ has an LR(0) item of the form
\begin{align*}
    N \rightarrow \alpha \bullet x \beta
\end{align*}
where $x$ is either a terminal symbol or a nonterminal symbol, then there is a goto state, 
$s_j$, from the state $s_1$ on $x$, and $s_j$ includes a kernel item of the form
\begin{align*}
    N \rightarrow \alpha x \bullet \beta
\end{align*}
If there are multiple items in $s_i$ with the same $x$ immediately to the right of the 
$\bullet$ then the goto state $s_j$ includes all those items but with the $\bullet$ after 
that occurence of $x$ rather than before it.

\subsubsubsubsection{Actions}
An LR(0) item of the form
\begin{itemize}
    \item $N \rightarrow \alpha \bullet a \beta$ where $a$ is a terminal symbol, indicates 
    the state containing the item has a shift  \textbf{parsing} action
    \item $S' \rightarrow S \bullet$ where $S'$ is the (introduced) start symbol for the 
    grammer, inidcates the state containing the item has an  \textbf{accept} action
    \item $N \rightarrow \alpha \bullet$ where $N$ is not the (introduced) start symbol for 
    the grammar, indicates the state containing the item has a parsing action 
    \textbf{reduce $N \rightarrow \alpha$}
\end{itemize}
A shift action at end-of-file is an error, as is an accept action when the input is not at 
end-of-file.

\subsubsubsubsection{Example}
\begin{equation*}
    \begin{aligned}[c]
        S \rightarrow A
    \end{aligned}
    \qquad 
    \begin{aligned}[c]
        A \rightarrow (\ A\ )
    \end{aligned}
    \qquad 
    \begin{aligned}[c]
        A \rightarrow a
    \end{aligned}
\end{equation*}
\image{LR(0) Automaton Example.png}
\begin{center}
    \begin{tabular}{|lr|l|}
        \hline
        Parsing stack \hspace{0.5cm} & Input & Parsing action \\
        \hline
        \$0 & ((a))\$ & shift \\
        \$0(2 & (a))\$ & shift \\
        \$0(2(2 & a))\$ & shift \\
        \$0(2(2a3 & ))\$ & reduce $A \rightarrow a$ \\
        \$0(2(2A4 & ))\$ & shift \\
        \$0(2(2A4)5 & )\$ & reduce \( A \rightarrow (A)\) \\
        \$0(2A4 & )\$ & shift \\
        \$0(2A4)5 & \$ & reduce \( A \rightarrow (A)\) \\
        \$0A1 & \$ & accept \\
        \hline
    \end{tabular}
\end{center}


\subsubsubsubsection{Parsing Conflicts}
If a state in an LR(0) parsing automaton has more than one action, 
there is a parsing action conflict.

A grammar is LR(0) if none of the states in its LR(0) parsing 
automaton contains a parsing action conflict.

\subsubsubsection{LR(1) Grammars}
An LR(1) parsing item is a pair
\begin{align*}
    [N \rightarrow \alpha \bullet \beta, T]
\end{align*}
consisting of
\begin{itemize}
    \item an LR(0) parsing item $N \rightarrow a \bullet \beta$, and
    \item a set $T$ of terminal symbols called a look-ahead set.
\end{itemize}
The above item indicates that in matchin $N$, $\alpha$ has been matched and $\beta$ is yet 
to be matched, in a context in which $N$ can be followed by a terminal symbol in the set $T$.

\subsubsubsubsection{Parsing Automaton}
The kernal item of the initial state is
\begin{align*}
    [S' \rightarrow \bullet S, \$]
\end{align*}
where $S$ is the start symbol of the grammar and we introduce a fresh replacement start 
symbol $S'$ and production $S' \rightarrow S$. This fresh production is used to determine 
when parsing has completed.

\subsubsubsubsection{Derived Items}
If a state has an LR(1) item of the form
\begin{align*}
    [N \rightarrow \alpha \bullet M \beta, T]
\end{align*}
where the nonterminal M has productions
\begin{align*}
    M \rightarrow \alpha_1 | \alpha_2 | \dots | \alpha_m
\end{align*}
and $T = \{a_1, a_2, \dots, a_n\}$, then the state also includes the derived items
\begin{align*}
    [N \rightarrow \bullet \alpha_1, T]
    \dots
    [N \rightarrow \bullet \alpha_m, T]
\end{align*}
where if $\beta$ is not nullable $T' = First(\beta)$ and if $\beta$ is nullable, 
$T' = First(\beta) - \{\epsilon\} \cup T$.

\subsubsubsubsection{Parsing Actions}
An LR(1) item of the form
\begin{itemize}
    \item $[N \rightarrow \alpha \bullet a \beta, T]$, where $a$ is a terminal symbol, 
    indicates the state containing the item has a \textbf{shift} parsion action if the
    next input $x$ is $a$
    \item $[S' \rightarrow \S \bullet, \$]$, where $S'$ is the added start symbol for 
    the grammar, indicates the state containing the item has an \textbf{accept} action 
    if there is no more input.
    \item $[N \rightarrow \alpha \bullet, T]$, where $N$ is not $S'$, indicates the state 
    containing the item has a parsing action \textbf{reduce $N \rightarrow \alpha$} if 
    the next input $x$ is in $T$.
\end{itemize}

\subsubsubsubsection{Example}
\begin{equation*}
    \begin{aligned}[c]
        S \rightarrow E
    \end{aligned}
    \qquad 
    \begin{aligned}[c]
        E \rightarrow E + n
    \end{aligned}
    \qquad 
    \begin{aligned}[c]
        E \rightarrow n
    \end{aligned}
\end{equation*}
\image{LR(1) Automaton Example.png}

\subsubsubsubsection{Parsing Action Conflicts}
If a state in an LR(1) parsing automaton has more than one action for a look-ahead 
terminal symbol, there is a parsing action conflict.

A grammar is LR(1) if none of the states in its LR(1) parsing automaton contains a parsing
action conflict.


\subsubsubsection{LALR(1) Parsing}
\begin{itemize}
    \item An LALR(1) parsing automaton can be formed from an LR(1) parsing automaton by 
    merging states that have identical sets of LR(0) items but possibly different look-ahead 
    sets.
    \item A grammar is LALR(1) if none of the states in its LALR(1) parsing automaton contains 
    a parsing action conflict.
\end{itemize}

\subsection{Left Factoring \& Left Recursion}

\subsubsection{Left Factoring Productions}
Not all EBNF grammars are suitable for Recursive-Descent Parsing, 
however, sometimes we can rewrite them into a form that is suitable.

\subsubsubsection{Left Factor Rewriting Rule}
To remove the left factor from
\begin{align*}
    A \rightarrow \alpha\ \beta\ |\ \alpha\ \gamma,
\end{align*}
we can rewrite the production using an aditional nonterminal $A'$ as
\begin{align*}
    A &\rightarrow \alpha\ A'\\
    A' &\rightarrow \beta\ |\ \gamma
\end{align*}
\subsubsection{Left Recursive Productions}
A production of the form
\begin{align*}
    E \rightarrow E\ +\ T\ |\ T
\end{align*}
is not suitable for RDP because the left recursion in the grammar leads to an infinite recursion.

\subsubsubsection{Immediate Left Recursion Rewriting Rule (Simple)}
To remove the left recursion from
\begin{align*}
    A \rightarrow A\ \alpha\ |\ \beta,
\end{align*}
we can rewrite the production as
\begin{align*}
    A &\rightarrow \beta\ A'\\
    A' &\rightarrow \epsilon\ |\ \alpha\ A'
\end{align*}

\subsubsubsection{Immediate Left Recursion Rewriting Rule (General)}
To remove the left recursion from the general case
\begin{align*}
    A \rightarrow A\alpha_1|A\alpha_2|\dots|A\alpha_n|\beta_1|\beta_2|\dots|\beta_m,
\end{align*}
we can use grouping to see the structure in the same form as the simple case
\begin{align*}
    A \rightarrow A(\alpha_1|\alpha_2|\dots|\alpha_n)|(\beta_1|\beta_2|\dots|\beta_m)
\end{align*}
which allows us to rewrite the production as follows,
\begin{align*}
    A &\rightarrow (\beta_1|\beta_2|\dots|\beta_m)A'\\
    A' &\rightarrow \epsilon|(\alpha_1|\alpha_2|\dots|\alpha_n)A'
\end{align*}

\subsubsubsection{Indirect Left Recursion Rewriting Rule}
The following productions have indirect recursion from $A \rightarrow B \rightarrow C \rightarrow A$.
\begin{align*}
    A &\rightarrow B\ \alpha\\
    B &\rightarrow C\ \beta\\
    C &\rightarrow A\ \gamma_1\ |\  \gamma_2
\end{align*}
To remove the recursion we can first collapse $B$ into $A$ as follows.
\begin{align*}
    A &\rightarrow C\ \beta\ \alpha\\
    C &\rightarrow A\ \gamma_1\ |\ \gamma_2
\end{align*}
Then we can collapse $C$ into $A$.
\begin{align*}
    A &\rightarrow (A\ \gamma_1\ |\  \gamma_2)\ \beta\ \alpha\\
    &\ \ \ \ \ \ \ \ \ \ \downarrow\\
    A &\rightarrow A\ \gamma_1\ \beta\ \alpha\ |\ \gamma_2\ \beta\ \alpha
\end{align*}
This now leaves a simple direct left recursion which we can remove as follows.
\begin{align*}
    A &\rightarrow \gamma_2\ \beta\ \alpha\ A'\\
    A' &\rightarrow \gamma_1\ \beta\ \alpha\ A'
\end{align*}


\subsubsection{Stack Organisation}
\subsubsubsection{Definition}
A stack machine consists of the following chunks of memory.
\begin{itemize}
    \item \textbf{stack} - the portion of memory used for both calculating the values of 
    expressions as well as storing activation records for every active procedure call.
    \item \textbf{heap} - the portion of memory for dynamic allocation of Objects.
    \item \textbf{code space} - the portion of memory where the machine instructions are 
    stored.
\end{itemize}
The stack and heap are stored in one contiguous area of memory. The stack grows from the 
bottom (address 0) and the heap grows down from the top.

The machine has four special registers:
\begin{itemize}
    \item \textbf{stack pointer} - contains the address for the top of the stack + 1
    \item \textbf{stack limit} - contains the address of the upper limit for the stack 
    and the bottom of the heap
    \item \textbf{frame pointer} - contains the address of teh stack frame for the 
    current procedure
    \item \textbf{program counter} - contains the address of the next machine instruction
\end{itemize}

\subsubsubsection{Procedures}
Each time a procedure is entered via a call, the machine must keep track of information 
such as the return address and the local variables for that call in a procedure frame. 
This frame is stored on the stack.

A pointer to the current procedures' frame is stored in a globally known location.
\subsubsubsubsection{Calling a Procedure}
\begin{itemize}
    \item parameters to the procedure are pushed to the stack
    \item a static link is pushed onto the stack
    \item the current frame pointer is pushed onto the stack to create the dynamic link
    \item the frame pointer is set so that it contains the address of the start of the 
    new stack frame
    \item the current value of the program counter is pushed onto the stack to form the 
    return address
    \item the program counter is set to the address of the procedure
    \item space is allocated on the stack for any local variables
\end{itemize}
\begin{center}
    \image[\columnwidth/2]{Stack Frame.png}
\end{center}

\subsubsubsubsection{Returning From a Procedure}
\begin{itemize}
    \item the program counter is set to the return address in the current activation record
    \item the frame pointer is set to the dynamic link
    \item the stack pointer is set so that all the space used by the stack frame (but not 
    parameters) is popped from the stack
    \item execution continues at the instruction addressed by the (restored) program counter
    \item after return, the calling procedure handles deallocating any parameters
\end{itemize}

\subsubsubsubsection{Local Variables}
Local variables are stored within the procedures frame. They are accessed by an offset 
relative to the frame pointer.

\subsubsubsubsection{Non-local Variables}
To allow access to variables outside of the enclosing procedure, the stack frame for a 
procedure includes a \textit{static link} which contains the address of the stack frame 
for the enclosing procedure (the procedure in which the enclosed procedure is defined).

To access the non-local variable $n$ from a procedure, we continuesly access the static link 
of the enclosing procedures until we are in the procedure in which $n$ is defined. We can then
add the offset to the variable $n$ to get the final address of the non-local variable.

\subsubsubsection{Parameters}
\begin{itemize}
    \item \textbf{Formal parameters} are the paremeters used in the declaration of the procedure 
    in its header.
    \item \textbf{Actual parameters} are the actual parameters passed to a procedure on a call.
\end{itemize}
\subsubsubsubsection{Call-by-value}
\begin{itemize}
    \item parameters are expressions that are evaluated and coerced to the type of the formal 
    paramter.
    \item the values of the parameters are loaded onto the stack as part of the calling sequence.
    \item Once the procedure is called and the stack frame has been established, the formal 
    parameters of the procedure can be accessed like local variables
    \item Accesses to the formal parameter access the location on the stack containing the value of the
    corresponding actual parameter.
\end{itemize}

\subsubsubsubsection{Call-by-const}
The formal parameter acts as a read-only local variable that is assigned the value of the actual 
parameter expression.

\subsubsubsubsection{Call-by-result}
The formal parameter acts as a local variable whose final value is assigned to the actual parameter 
variable.

\subsubsubsubsection{Call-by-value-result}
A single parameter acts as both a value and a result parameter.

\subsubsubsubsection{Call-by-reference}
The formal parameter is the address of the actual parameter variable. All references to the formal 
parameter are applied to the actual parameter variable immediately.

\subsubsubsubsection{Call-by-sharing}
The same as call-by-value, but what is passed is a reference to an object (e.g. Java) rather than 
the values of the object.

\subsubsubsubsection{Call-by-name}
The actual parameter expression is evaluated every time the formal parameter is accessed.

\subsubsubsubsection{Passing Procedures as Parameters}
The address of the procedure as well as the static link for the procedure's environment is passed.

\subsubsubsection{Function Results}
\begin{itemize}
    \item A function can return a result that can be used as part of an expression.
    \item The result of a function call should be left on top of the stack after the stack frame 
    is removed.
    \item To ensure the returned value is before the stack frame, free space is allocataed for 
    the result before the parameters are loaded onto the stack and the frame is set up.
\end{itemize}

\subsubsubsubsection{Returning Procedures}
Return the address of the procedure as well as the static link for the procedures environment.
This requires the environment of the returned procedure to be maintained which makes the simple
stack-based allocation of frames insufficient.

\subsubsubsection{Variable Aliasing}
\subsubsubsubsection{Parameter Aliasing}
In languages with call-by-reference, the same variable can be passed to two (or more) different
parameters leading to variable aliasing.

\subsubsubsubsection{Global Variable Aliasing}
If a variable is passed as a reference parameter to a procedure that can access the same variable
as a global variabel, then within the procedure there are two aliases for the same variable.

\subsubsubsection{Pointer Aliasing}
\subsubsubsubsection{Parameter Aliasing}
In languages with call by sharing, the same reference to an object can be passed to two different 
parameters leading to one having two aliases for the same reference.

\subsubsubsubsection{Global Variable Aliasing}
If a reference that is passed as a parameter to a procedure is also directly accessible as a 
global variable from the procedure. this leads to the procedure having two aliases for the one 
reference.

\subsubsection{Objects}
Each field in an object is at a fixed offset from the start of 
every instance in the heap.

\subsubsubsection{Subclassing}
Aditional fields follow the inherited fields \& have fixed 
offsets from the start. Fields that are redeclared replace the 
previously declared field at the same offset.

\subsubsubsubsection{Dynamic Dispatch Table}
\begin{itemize}
    \item it has an entry for every method of the class (including inherited 
    methods)
    \item the entry gives the address of the code for the method
    \item the entries are at a fixed offset from the start of the dynamic 
    dispatch table
    \item the entries for inherited \& overridden methods are at the same 
    offsets as in the superclass
\end{itemize}
\image[\columnwidth]{Dynamic Dispatch Table.png}

\subsubsubsubsection{This}
\textbf{this} is passed as an additional implicit parameter to each method.

\subsubsubsubsection{Static Fields \& Methods}
Static fields \& methods are resolved statically (at compile time). Static
methods are not called on an object \& hence they do not have an object 
reference as an implicit this parameter.

\subsubsection{Heap Organisation}
\subsubsubsubsection{Issues w/ Explicit Deallocation}
\begin{itemize}
    \item \textbf{Dangling references} --- an object can be freed via one reference but 
    still be accessible via other references.
    \item \textbf{Memory leaks} --- all references to an object are removed but the object 
    was not freed.
    \item \textbf{Memory fragmentation} --- the memory consists of alternating alloctated 
    \& free areas, with no large area of free memory.
    \item \textbf{Locality of reference} --- how spread out the memory is in the address space.
\end{itemize}

\subsubsubsection{Garbage Collection}
An object is accessible if it can be reached either
\begin{itemize}
    \item directly via a reference in a global or local variable, or
    \item indirectly via a reference in an object that is accessible.
\end{itemize}
Garbage collection determined which objects are not accessible \& recovers the space used 
by them.

\subsubsubsubsection{Mark-\&-sweep}
\begin{itemize}
    \item a phase that \textbf{marks} all the accessible objects
    \item a phase that \textbf{sweeps} up the objects left unmarked \& adds them to the 
    free list
\end{itemize}

\subsubsubsubsection{Stop-\&-copy}
\begin{itemize}
    \item divides the available memory into two spaces
    \item memory is allocated sequentially from one space until it runs out
    \item garbage collection consists of relocating all accessible objects from the first 
    space to the second space
\end{itemize}
Overcomes memory fragmentation problems, but copying can be expensive in time.

\subsubsubsubsection{Generational Schemes}
Generational schemes use a scheme similar to the stop-\&-copy scheme, but make us of more 
spaces
\begin{itemize}
    \item the spaces are organised based on the length of time its objects have survived
    \item older objects are migrated to an old object space \& newer objects go in the a 
    new object space
    \item the newer the space, the more frequently it is garabage collected
\end{itemize}

\subsubsubsubsection{Issues w/ Garbage Collection}
\begin{itemize}
    \item Garbage collection has a greater time overhead than explicit deallocation
    \item Response time can vary because garbage collection can happen at any time 
    \& is time expensive
    \item Real-time response hard to guarantee
\end{itemize}

\subsubsubsubsection{On-the-fly}
\begin{itemize}
    \item Garbage collection process takes place incrementally, interleaed with 
    execution of the program.
    \item Each time an object is allocated, the garbage collector can do a bit 
    of work.
\end{itemize}

\subsubsubsubsection{Concurrent}
\begin{itemize}
    \item The garbage collector runs concurrently with the program on a separate 
    processor.
\end{itemize}

\subsection{Regular Expressions}
\subsubsubsection{Deterministic Finite Automaton (DFA)}
A DFA, $D$, consists of
\begin{itemize}
    \item A finite alphabet of symbols, $\Sigma$
    \item A finite set of states, $S$
    \item A transition function, $T : \Sigma \times S \rightarrow S$, which maps an (input) symbol \& a (current) state to the (next) state; the function $T$ may not be defined for all pairs of symbols \& state. 
    \item A start state, $s_0$ 
    \item A set of final (or accepting) states, $F$
\end{itemize}

\image[100px]{DFA.png}

\subsubsubsection{NFA}

\subsubsubsection{From an Regex to a NFA}
In the translation from a regular expression, $r$, to an NFA, the generated NFA has a few properties that do not necessarily hold for an arbitrary NFA (i.e. one not generated from a regular expression).
\begin{itemize}
    \item The NFA has a single final (accepting) state.
    \item The initial state of the NFA only has outgoing transitions.
    \item The final state only has incoming transitions.
\end{itemize}
The translation rules preserve these properties.

\subsubsubsection{From NFA to DFA}
A DFA cannot have 
\begin{itemize}
    \item more than one transition leaving a state on the same symbol
    \item any empty transitions
\end{itemize}
An NFA N can be translated to an equivalent DFA D.
\begin{itemize}
    \item equivalent in the sense that they accept the same languages, i.e., $\pazocal{L}(N) = \pazocal{L}(D)$.
\end{itemize}

\subsubsubsection{From NFA to DFA}
An NFA is transformed to a DFA in which the labels on the states of the DFA are sets of states from the NFA.
The sets of states that label a DFA state are formed by collecting all the states that can be reached from NFA states by empty transitions. 

\subsubsubsection{Empty Closure of a state}
Empty Closure of a state
The empty closure of a state $x$ in an NFA $N$, $\epsilon$-closure$(x, N)$, is
the set of states in $N$ that are reachable from x via any number
of empty transitions

\subsubsubsection{Empty Closure of a set of states}
The empty closure of a set of states $X$ in an NFA $N$, $\epsilon$-closure$(X , N)$, is the set of states in $N$ that are reachable from any of the states in $X$ via any number of empty transitions. 

\subsubsection{From NFA to DFA}

The following process is repeated until there are no unmarked DFA states left:
\begin{itemize}
    \item An unmarked DFA state $S$ is selected (the first one is $S_0$).
    \item For each symbol $a$,
    \begin{itemize}
        \item we consider the set of states that can be reached from any state in $S$ by a transition on $a$; call this set of states $X$.
        \item If $X$ is nonempty, we add a new state to the DFA labeled with $X' = \epsilon\text{-closure}(X, N)$, unless a state with that label already exists, in which case it is reused.
        \item A transition from $S$ to $X'$ on $a$ is added to the DFA.
    \end{itemize}
    \item The state $S$ is marked as having been processed.
\end{itemize}

\subsubsubsection{Minimising a DFA}
To minimise the DFA we merge states that have the same
transition to the equivalent states. For the example, A, B \& C
are equivalent.


\scriptsize

\end{multicols}
\end{document}
