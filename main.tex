\documentclass[10pt,landscape]{article}
\usepackage{multicol}
\usepackage{calc}
\usepackage{ifthen}
\usepackage[landscape]{geometry}
\usepackage{hyperref}
\usepackage{amsmath}
\usepackage{multicol}
\usepackage{fancyvrb}
\newcommand\userinput[1]{\textbf{#1}}

% To make this come out properly in landscape mode, do one of the following
% 1.
%  pdflatex latexsheet.tex
%
% 2.
%  latex latexsheet.tex
%  dvips -P pdf  -t landscape latexsheet.dvi
%  ps2pdf latexsheet.ps


% If you're reading this, be prepared for confusion.  Making this was
% a learning experience for me, and it shows.  Much of the placement
% was hacked in; if you make it better, let me know...


% 2008-04
% Changed page margin code to use the geometry package. Also added code for
% conditional page margins, depending on paper size. Thanks to Uwe Ziegenhagen
% for the suggestions.

% 2006-08
% Made changes based on suggestions from Gene Cooperman. <gene at ccs.neu.edu>


% To Do:
% \listoffigures \listoftables
% \setcounter{secnumdepth}{0}


% This sets page margins to .5 inch if using letter paper, and to 1cm
% if using A4 paper. (This probably isn't strictly necessary.)
% If using another size paper, use default 1cm margins.
\ifthenelse{\lengthtest { \paperwidth = 11in}}
	{ \geometry{top=.15in,left=.10in,right=.10in,bottom=.15in} }
	{\ifthenelse{ \lengthtest{ \paperwidth = 297mm}}
		{\geometry{top=1cm,left=1cm,right=1cm,bottom=1cm} }
		{\geometry{top=1cm,left=1cm,right=1cm,bottom=1cm} }
	}

% Turn off header and footer
\pagestyle{empty}
 

% Redefine section commands to use less space
\makeatletter
\renewcommand{\section}{\@startsection{section}{1}{0mm}%
                                {-1ex plus -.5ex minus -.2ex}%
                                {0.5ex plus .2ex}%x
                                {\normalfont\large\bfseries}}
\renewcommand{\subsection}{\@startsection{subsection}{2}{0mm}%
                                {-1explus -.5ex minus -.2ex}%
                                {0.5ex plus .2ex}%
                                {\normalfont\normalsize\bfseries}}
\renewcommand{\subsubsection}{\@startsection{subsubsection}{3}{0mm}%
                                {-1ex plus -.5ex minus -.2ex}%
                                {1ex plus .2ex}%
                                {\normalfont\small\bfseries}}                   
\newcommand{\subsubsubsection}{\@startsection{subsubsection}{4}{0mm}%
                                {-1ex plus -.5ex minus -.2ex}%
                                {1ex plus .2ex}%
                                {\normalfont\footnotesize\bfseries}}
\makeatother
\newcommand{\imp}{\Rightarrow}

% Define BibTeX command
\def\BibTeX{{\rm B\kern-.05em{\sc i\kern-.025em b}\kern-.08em
    T\kern-.1667em\lower.7ex\hbox{E}\kern-.125emX}}

% Don't print section numbers
\setcounter{secnumdepth}{0}


\setlength{\parindent}{0pt}
\setlength{\parskip}{0pt plus 0.5ex}

\newcommand\tab[1][0.5cm]{\hspace*{#1}}

% -----------------------------------------------------------------------

\begin{document}

\raggedright
\fontsize{10pt}{0.8pt}\selectfont
\begin{multicols}{3}


% multicol parameters
% These lengths are set only within the two main columns
%\setlength{\columnseprule}{0.25pt}
\setlength{\premulticols}{1pt}
\setlength{\postmulticols}{1pt}
\setlength{\multicolsep}{1pt}
\setlength{\columnsep}{2pt}

\begin{center}
     \Large{\textbf{CSSE3100 Crib Sheet}} \\
\end{center}

\section{Exam Format}
The confirmed format of the exam is:
\begin{tabular}{@{}ll@{}}
\verb!Q1!    & weakest precondition reasoning. \\
\verb!Q2!  & method specification and loop invariants. \\
\verb!Q3! & recursion and termination metrics. \\
\verb!Q4!  & classes and data structures. \\
\verb!Q5!  & lemmas and functional programming
\end{tabular}

This section will be removed before the exam

\section{Question 1}

\subsection{Predicate Logic}

\begin{tabular}{@{}ll@{}}
\fontsize{5pt}{6pt}\selectfont
$ A \land (A \lor B) \equiv A \equiv A \lor (A \land B)$ & \verb! (A.6)! \\
$ A \land (B \lor C) \equiv (A \land B) \lor (A \land C)$ & \verb! (A.7)! \\
$ A \lor (B \land C) \equiv (A \lor B) \land (A \lor C)$ & \verb! (A.8)! \\
$ \neg (A \land B) \equiv \neg A \lor \neg B$ & \verb! (A.18)! \\
$ \neg (A \lor B) \equiv \neg A \land \neg B$ & \verb! (A.19)! \\
$ A \lor (\neg A \land B) \equiv A \lor B$ & \verb! (A.20)! \\
$ A \land (\neg A \lor B) \equiv A \land B$ & \verb! (A.21)! \\
$ A \imp B \equiv \neg A \lor B$ & \verb! (A.22)! \\
$ A \imp B \equiv \neg(A \land \neg B)$ & \verb! (A.24)! \\
$ \neg(A \imp B) \equiv A \land \neg B$ & \verb! (A.25)! \\
$ A \imp B \equiv \neg B \imp \neg A$ & \verb! (A.26)! \\
$ C \imp (A \land B) \equiv (C \imp A) \land (C \imp B)$ & \verb! (A.33)! \\
$ (A \lor B) \imp C \equiv (A \imp C) \land (B \imp C)$ & \verb! (A.34)! \\
$ C \imp (A \lor B) \equiv (C \imp A) \lor (C \imp B)$ & \verb! (A.35)! \\
$ (A \land B) \imp C \equiv (A \imp C) \lor (B \imp C)$ & \verb! (A.36)! \\
$ A \imp (B \imp C) \equiv (A \land B) \imp C \equiv$ & \verb! (A.37)! \\
$ B \imp (A \imp C)$ & \\
$ (A \imp B) \land (\neg A \imp C) \equiv$ & \verb! (A.38)! \\
$ (A \land B) \lor (\neg A \land C)$ & \\
$ (\forall x \text{ s.t. } x = E \imp A) \equiv A[x \backslash E] \equiv $ & \verb! (A.56)! \\
$ (\exists x \text{ s.t. } x = E \land A)$ & \\
$ \forall x :: A \land B = (\forall x :: A) \land (\forall x :: B)$ & \verb! (A.65)! \\
$ \forall x :: A = A \text{ provided } x \text{ not free in } A$ & \verb! (A.74)! \\
\end{tabular}

\subsection{Rules to know}

\subsubsection{Basic Function} 
\begin{verbatim}
method MyMethod(x: int) returns (y: int)
    requires x == 10
    ensures y >= 25
{
    {x == 10}
    {x + 3 + 12 == 25}
    var a := x + 3;
    {a + 12 == 25}
    var b := 12;
    {a + b == 25}
    y := a + b;
    {y >= 25}
}

\end{verbatim}


\subsubsection{Loops} 
\begin{verbatim}
{J}
while B
        invariant J
{
        {B && J}
        ... 
        {J}
}
{J && !B}
\end{verbatim}

\begin{verbatim}
{y >= 4 && z >= x}
while z < 0
        invariant y >= 4 && z >= x
{
        {z < 0 && y >= 4 && z >= x}
        {y >= 4 && z + y >= x}
        z := z + y;
        {y >= 4 && z >= x}
}
{z >= 0 && y >= 4 && z >= x}
\end{verbatim}

\subsubsection{Arrays} 
\begin{verbatim}
var a := new string[20];
# Type of a is array<string>
var m := new bool[3, 10];
# Type of m is array2<bool>

\end{verbatim}
\begin{verbatim}
method LinearSearch<T>(a: array<T>, P: T -> bool)
returns (n: int)
ensures 0 <= n <= a.Length
ensures n == a.Length || P(a[n])
ensures n == a.Length ==>
forall i :: 0 <= i < a.Length ==> !P(a[i])
{
n := 0;
while n != a.Length
        invariant 0 <= n <= a.Length
        invariant forall i :: 0 <= i < n ==>
                        !P(a[i])
{ 0 <= n < a.Length &&
(!P(a[n]) ==> (forall i :: 0 <= i < n ==>
                        !P(a[i])) 
                                && !P(a[n])) }
{ (P(a[n]) ==> 0 <= n <= a.Length &&
(n == a.Length || P(a[n])) &&
(n == a.Length ==>
forall i :: 0 <= i < a.Length ==> !P(a[i]))) &&
(!P(a[n]) ==> (forall i :: 0 <= i < n ==> 
                        !P(a[i])) && !P(a[n])) }
if (P(a[n])) {
        return;
}
{ (forall i :: 0 <= i < n ==> !P(a[i])) (A.56)
&& (forall i :: i == n ==> !P(a[i])) } (A.65)
{ forall i :: (0 <= i < n ==> !P(a[i])) &&
                        (i == n ==>
                        !P(a[i])) } (A.34)
{ forall i :: 0 <= i < n || i == n ==> !P(a[i])}
{ forall i :: 0 <= i < n + 1 ==> !P(a[i]) }
n := n + 1;
{ forall i :: 0 <= i < n ==> !P(a[i]) }
\end{verbatim}

\subsubsection{Methods} 
\begin{verbatim}
wp(t := M(E), Q )
  = P[x\E]
    &&  forall y' ::
      R[x,y\E, y'] 
        ==> Q[t\y']
\end{verbatim}
\begin{verbatim}
Given:
method Triple(x: int) returns (y: int)
requires x >= 0
ensures y == 3*x {}


{ u == 15}
{ u + 3 >= 0 &&
        3*(u + 3) == 54 } (A.56)
{ u + 3 >= 0 &&
        forall y' :: y' == 3*(u + 3)
                ==> y' == 54 }
t := Triple(u + 3);
{ t == 54 }

\end{verbatim}
\begin{verbatim}
function SeqSum(s: seq<int>, lo: int, hi: int): int
requires 0 <= lo <= hi <= |s|
decreases hi - lo
{
        if lo == hi then 0 else s[lo] +
            SeqSum(s, lo + 1, hi)
}
\end{verbatim}

\section{Question 2}
\subsection{Loop Design Techniques}
\subsubsection{Look in the postcondition.}
For a postcondition A \&\& B,
choose the invariant to be A and the guard to be !B.
\begin{verbatim}
method SquareRoot(N: nat) returns (r: nat)
ensures r*r <= N && N < (r + 1)*(r + 1)
    { { 0 <= N }
    { 0*0 <= N}
    r := 0;
    { r*r <= N }
    while (r + 1)*(r + 1) <= N
    invariant r*r <= N
    {
        { (r + 1)*(r + 1) <= N 
                && r*r <= N } (strengthen)
        { (r + 1)*(r + 1) <= N }
        r := r + 1;
        { r*r <= N }
    }
}
\end{verbatim}

\subsubsection{Programming by wishing}
If a problem can be made simpler by having a
precomputed quantity Q, then introduce a new
variable q with the intention of establishing and
maintaining the invariant q == Q

\begin{verbatim}
method SquareRoot(N: nat) returns (r: nat)
ensures r*r <= N < (r + 1)*(r + 1)
{
    r := 0;
    var s := 1;
    while s <= N
    invariant r*r <= N
    invariant s == (r + 1)*(r + 1)
    {
        s := s + 2*r + 3;
        r := r + 1;
    }
}
\end{verbatim}

\subsubsection{Replace a constant by a variable}
For a loop to establish a condition P(C), where C is an
expression that is held constant throughout the loop,
use a variable k that the loop changes until it equals C,
and make P(k) a loop invariant.

For example, Min method (Week 4) had postcondition
\begin{verbatim}
    ensures forall i :: 0 <= i < a.Length ==>
                                        m <= a[i]
\end{verbatim}
and invariant
\begin{verbatim}
    invariant forall i :: 0 <= i < n ==> m <= a[i]
\end{verbatim}

\subsubsection{What's yet to be done}.
If you're trying to solve a problem of the form
p == F(n), replacement of a constant by a variable
results in a what-has-been-done invariant
\begin{verbatim}
    invariant p == F(i)
\end{verbatim}
Alternatively, you may use a what's-yet-to-be-done
invariant

\begin{verbatim}
    invariant p @ F(n – i) == F(n)
\end{verbatim}
where @ is some kind of combination operation.

\subsubsection{Use the postcondition}
To establish a postcondition Q, make Q a loop
invariant.

For the Min example, to ensure the postcondiVon
\begin{verbatim}
ensures exists i :: 0 <= i < a.Length && m == a[i]
\end{verbatim}
we used the invariant
\begin{verbatim}
invariant exists i :: 0 <= i < a.Length && m == a[i]
\end{verbatim}

\section{Question 3}
\subsection{Termination Metrics}
Any set of values which have a \textit{well-founded} order can be used as a termination metric.

An order $\succ$ is well-founded when
\begin{itemize}
    \item $\succ$ is irreflexive: a $\succ$ a never holds
    \item $\succ$ is transitive:\\
        \tab a $\succ$ b \&\& b $\succ$ c $\implies$ a $\succ$ c
    \item there is no infinite descending chain\\
        \tab $\text{a}_1 \succ \text{a}_2  \succ \text{a}_3  \succ \dots$
\end{itemize}

We write X decreases to x as $X \succ x$.

For integers, $X \succ x$ when X > x \&\& X >= 0. \\
For booleans, $X \succ x$ when X \&\& !x.

A termination metric for a recursive function is a metric that can be proven to decrease every iteration.

E.g. for the function;
\begin{verbatim}
    function F(x: int): int 
    { 
        if x < 10 then x else F(x – 1)
    }
\end{verbatim}
the termination metric would be x since $x \succ x - 1$.

\subsubsection{Lexicographic tuples}
A lexicographic order is a component-wise comparison where earlier components are more significant.

$\{\text{a}_0, \text{a}_1, \text{a}_2, \dots, \text{a}_n\} \succ \{\text{b}_0, \text{b}_2, \text{b}_3, \dots, \text{b}_n\}$ if and only if\\
$\text{a}_0 \succ \text{b}_0\ ||\ (\text{a}_0 == \text{b}_0$ \&\& $\text{a}_1 \succ \text{b}_1)\ ||$\\
$\tab (\text{a}_0 == \text{b}_0$ \&\& $\text{a}_1 == \text{b}_1$ \&\&\\
$\tab \tab \text{a}_2 \succ \text{b}_2)\ || \dots\ ||$\\
$\tab (\text{a}_0 == \text{b}_0$ \&\& $\text{a}_1 == \text{b}_1$ \&\& $\dots$ \&\&\\
$\tab \tab \text{a}_{n-1} == \text{b}_{n-1}$ \&\& $\text{a}_n \succ \text{b}_n)$

A lexicographic ordering allows tuples to be used as termination metrics. 

\subsubsection{Mutually Recursive Functions}
Tuples can be used to provide termination metrics for mutually recursive functions since you can provide multiple values that the functions may reduce on.

E.g. for the following methods;
\begin{verbatim}
    method F(i: nat) returns (r: nat) { 
        if i <= 2 { r := 1; } 
        else {
            var h := H(i - 2);
            r := 1 + h; 
        } 
    } 

    method H(i: nat) returns (r: nat) { 
        if i == 0 { r := 0; } 
        else { 
            var f := F(i); 
            var h := H(i - 1); 
            r := f + h; 
        } 
    }
\end{verbatim}
the termination matrix would be \{i, 1\} for H and \{i, 0\} for F since the call F(i) in H will reduce on 1 $\succ$ 0. 

\section{Question 4}
\subsection{Classes}
Ghost variables can be used for specification and reasoning only.
\begin{Verbatim}[commandchars=\\\{\}]
    ghost var \textit{d}: \textit{T}
\end{Verbatim}

\subsubsection{Simple Classes}
A simple class consits of only simple object, (i.e. objects that are not stored on the heap).

The specification for a simple class consists of:
\begin{itemize}
    \item ghost variables for abstract state
    \item have class invariant, \textbf{ghost predicate Valid()}
    \item Valid() and functions have \textbf{reads this}
    \item constructor has \textbf{ensures Valid()}
    \item methods have \textbf{requires Valid()}, \textbf{modifies this}, \textbf{ensures Valid()}
\end{itemize}

Concrete states that consist of only simple objects are created and are related to the abstract state in \textbf{valid()}.

The constructor, methods, and functions must satisfy the class specification and will require both concrete and abstract state to be updated.

\subsubsection{Complex Classes}
Complex classes consist of any combination of simple and complex objects, (i.e. objects that are stored on the heap).

Complex classes require a representation set,
\begin{Verbatim}[commandchars=\\\{\}]
    \textbf{ghost var} Repr: set<object>
\end{Verbatim}

\subsubsubsection{Invariant}
The invariant valid will consist of the following, where a, a0, a1 are non-composite objects or arrays and b, b0, b1 are composite objects.
\begin{Verbatim}[commandchars=\\\{\}]
    \textbf{ghost predicate} Valid()
        \textbf{reads} this, Repr
        \textbf{ensures} Valid() ==> this \textbf{in} Repr
    \{ 
        this \textbf{in} Repr && ...
    \}
\end{Verbatim}

For a non-composite object or array \textbf{a}, include;
\begin{Verbatim}[commandchars=\\\{\}]
    a \textbf{in} Repr && a.Valid()
\end{Verbatim}

For a non-composite objects or arrays \textbf{a0, a1}, include;
\begin{Verbatim}[commandchars=\\\{\}]
    a0 != a1
\end{Verbatim}

For a composite object \textbf{b}, include;
\begin{Verbatim}[commandchars=\\\{\}]
    b \textbf{in} Repr && b.Repr <= Repr && 
    this !\textbf{in} b.Repr && b.Valid()
\end{Verbatim}

For a composite objects \textbf{b0, b1} and non-composite objects and arrays \textbf{a0, a1}, include;
\begin{Verbatim}[commandchars=\\\{\}]
    \{a0, a1\} !! b0.Repr !! b1.Repr  
\end{Verbatim}

\subsubsubsection{Constructor}
For a non-composite array or object \textbf{a} and a composite object \textbf{b}.
\begin{Verbatim}[commandchars=\\\{\}]
    \textbf{constructor}()
        \textbf{ensures} Valid() && \textbf{fresh}(Repr)
    \{
        ... \textit{(initialise concrete and abstract state)}
        \textbf{new};
        Repr := \{this, a, b\} + b.Repr;
    \}
\end{Verbatim}

\subsubsubsection{Functions}
\begin{Verbatim}[commandchars=\\\{\}]
    \textbf{function} F(x:X): Y()
        \textbf{requires} Valid()
        \textbf{reads} Repr
        \textbf{ensures} F(x) == \dots
\end{Verbatim}

\subsubsubsection{Methods (Mutating)}
\begin{Verbatim}[commandchars=\\\{\}]
    \textbf{method} M(x:X) returns Y()
        \textbf{requires} Valid()
        \textbf{modifies} Repr
        \textbf{ensures} Valid() && \textbf{valid}(Repr - \textbf{old}(Repr))
\end{Verbatim}

\section{Question 5}

\subsection{Lemmas}
\textbf{lemma} \textit{name}$(x_1: \textit{T}, x_2: \textit{T}, \ldots, x_n: \textit{T})$ \\
        \tab \textbf{requires} P \\
        \tab \textbf{ensures} R \\
\{ \}

Lemmas can be called in a method to \textbf{prove} the lemmas property from that point onwards.

\subsubsection{Weakest Precondition}
\textbf{wp}(M(E), Q)  =  \verb!P[x\E] && (R[x\E] ==> Q)!

\subsubsection{Calc}
To prove a lemma by hand, you can add a \textbf{calc} section into the lemmas body, where \textit{$\gamma$} is the default transitive operator between lines. \\
$\textbf{calc} \, \textit{$\gamma$} \, \{$ \\
$\tab    5*(x + 3);$ \\
$\tab == 5*x + 5*3;$ \\
$\tab == 5x + 15;$ \\
$\}$

You can use use any transitive operator between lines (e.g. $==>$). If no default operator is specific, the default is $==$.

The \textbf{calc} statements can also be added inline within a method instead of creating and calling a lemma.

\subsubsection{Induction}
Lemmas can also be used to prove using induction by recursively calling the lemma in the body. E.g. \\
\textbf{lemma} SumLemma(a: \textit{array<int>}, i: \textit{int}, j: \textit{int}) \\
        \tab \textbf{requires} P \\
        \tab \textbf{ensures} R \\
\{ \\
        \tab if i == j \{\} // base case: Dafny can prove\\
        \tab else \{\\
        \tab    \tab SumLemma(a, i+1, j); // inductive case\\
        \tab \}\\
\}

\subsection{Functional Programming}
Key features:
\begin{itemize}
        \item Program structures as mathematical functions
        \item Data is immutable (i.e. no heap, no side effects)
\end{itemize}

\subsubsection{Match}

\textbf{Match} is dafny's version of a switch statement, but it must cover all cases.\\
\tab \textbf{match} $x$\\
\tab \textbf{case} $c_1$\\
\tab \textbf{case} $c_2$\\
\tab \ldots\\
\tab \textbf{case} $c_n$\\

\subsubsection{Descriminators}
Discriminators can be used to check if a variable is a given type. E.g. xs.Nil\textbf{?} checks if xs is type Nil.

\subsubsection{Destructors}
Destructors are used to access data in a composite datatype. E.g. for a variable xs of the datatype\\
\textbf{datatype} \verb!List<T>! = Nil | Cons(head: T, tail: \verb!List<T>!),\\
head can be accessed using xs.head. Similarly tail can be accessed using xs.tail.

\subsubsection{Instrinsic vs Extrinsic Property}
\begin{itemize}
        \item An intrinsic property is a property defined within a specification.
        \item An extrinsic property is a property defined externally using a lemma.
        \item Methods in Dafny are opaque, so all properties in the specification are intrinsic.
        \item Functions are transparent, so properties can be intrinsic or extrinsic.
        \item Intrinsic properties are available every time we apply a function, whereas extrinsic properties are only available if we call the lemma.
        \item Having all properties exposed instrinsicly can lead to long verification times, so only define properties intrinsicly if they will be required for all applications of the function.
\end{itemize}


\scriptsize

\end{multicols}
\end{document}
