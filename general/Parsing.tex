\subsection{Parsing Theory}

\subsubsection{Context Free Grammars}

Basic Example of Context Free Grammar

$E \rightarrow E \ Op \ E $ \\
$E \rightarrow "(" E ")"$ \\
$E \rightarrow number$ \\
$Op \rightarrow "+"$ \\
$Op \rightarrow "-"$ \\
$Op \rightarrow "*"$ \\
Has start symbol $E$, nonterminals $\{E, Op\}$, and terminals
$\{“(”, “)”, number , “+”, “-”, “*”\}$

A context-free grammar consists of:
\begin{itemize}
\item A finite set, $\sum$,  of terminal symbols.
\item A finite nonempty set of nonterminal symbols (disjoint from the terminal symbols).
\item A finite nonempty set of producions of the form of $A \rightarrow \alpha$, where $A$ is a nonterminal symbol, and $\alpha$ is a possibly empty sequence of symbols, each of which is either a terminal of nonterminal symbol.
\item A start symbol that must be a nonterminal symbol
\end{itemize}

\subsubsubsection{Directly Derives}
If there is a production in the form of $N \rightarrow \gamma$ then we can directly derive $\alpha \ N \beta \ \rightarrow \alpha \ \gamma \ \beta$, where $\alpha$ and $\beta$ are possibly empty sequences of terminal and nonterminal symbols.

\subsubsubsection{Derives}
Given a sequence of terminal and nonterminal symbols, $\alpha$, derives a sequence $\beta$, written $\alpha \stackrel{*}{\Rightarrow} \beta$ if there is a finitie sequence of zero or more direct derivation steps that start from $\alpha$ and finishing with $\beta$, there must be one or more sequence $\gamma_1, \gamma_2, \ldots, \gamma_n$ such that $\alpha = \gamma_0 \Rightarrow \gamma_1 \Rightarrow \ldots \Rightarrow \gamma_n = \beta$. Note that zero steps are allowed.

\subsubsubsection{Nullable}
A possibly empty sequence of symbols, $\alpha$, is nullable if  $\alpha \stackrel{*}{\Rightarrow} \epsilon$ or $\alpha \stackrel{*}{\Rightarrow}$
Nullable rules
\begin{itemize}
        \item $\epsilon is nullable$
        \item any terminal symbol is not nullable
        \item a sequence of symbols is nullable if all of its constructs are nullable
        \item a set of alternatives is nullable if any of its constructs are nullable
        \item EBNF constructs for optionals and repetitions are nullable
        \item a nonterminal is nullable if there is a production with a nullable right-hand side
\end{itemize}

\subsubsubsection{Language}
The formal language $\pazocal{L}(G)$ corresponding to a Grammar, G, is:
$\pazocal{L}(G) = \{t \in seq \sum | S \stararrow t \}$

where S is the start symbol of $G$ and $\sum$ is its set of terminal symbols.

\subsubsubsection{Sentences and Sentenial Form}
A sequence of terminal symbols $t$ such that $S \stararrow t$ is called a \emph{sentence} of the language.

\subsubsubsection{Ambiguous grammars for sequence}
A grammar, \emph{G}, is ambiguous for a sentence, \emph{t}, in $\pazocal{L}$, if there is more than one parse tree for tree for \emph{t}.
\subsubsubsection{Ambiguous grammars}
A grammar, \emph{G}, is ambiguous if it is ambiguous for any setence in $\pazocal{L}(G)$.
\subsubsubsection{Left and right associative operators}
To remove the ambiguit and treat \emph{"-"} as  a left associate operator (as usual) we can rewrite the grammar to

$E \rightarrow E \ "-" \ T$ \\
$E \rightarrow T$ \\
$T \rightarrow N$ \\
and to treat \emph{"-"} as right associative we use
$E \rightarrow T "-" E$ \\
$E \rightarrow T$ \\
$T \rightarrow N$ \\