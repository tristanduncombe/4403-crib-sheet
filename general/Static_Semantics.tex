\subsection{Static Semantics}

\subsubsection{Types}
\subsubsubsection{Function Type}
$T_1 \rightarrow T_2$ indicates a mapping from an argument type $T_1$ to a result type $T_2$.

\subsubsubsection{Product Type}
$T_1 \times T_2$ is the type consisting of all pairs with first element of type $T_1$ and 
second element of type $T_2$.

\subsubsubsection{Reference Types}
A variable $x$ is a refererence to the value of $x$, hence we use $ref(T)$ which can be 
implicitly dereferences with Rule 3.6.

\subsubsection{Well-typed Expressions}
Expressions are defined in terms of integers, identifies, arithmetic and relational operators, 
and sub-expressions.

\subsubsubsection{Type Inferences}
Because expressions may contain identifiers, we need the context of a symbol table (represented 
by $syms \in \textbf{id} \rightarrow SymEntry$).
The domain of the symbol table is $\textbf{dom}(syms)$.

To state that the expression \textbf{e} is well typed in the context of the symbol table and has 
type $T$, we write $syms \vdash \textbf{e} : T$.

The rules are of the form of a possibly empty list of premises (above a horizontal line) followd by 
a consequent type inference below the line.

\subsubsubsection{Type Coercion}
We can write rules which allow us to implicitly coerce types from $T_1$ to $T_2$ if required. For 
example, the dereference rule (Rule 3.6).

\subsubsubsection{Type Derivations}
An expression is well typed provided one can apply one of the type inference rules to deduce its
type and all of the presmises of the infrerence rule are valid. If any premises are type inferences
then they must be shown to be valid in a similar fashion.

\subsubsubsection{Well-formed statements}
Statements are defined in terms of expressions and sub-statements. Whether or not a statement is well 
formed depends on the context of the symbol table.
\begin{gather*}
    syms \vdash WFStatement(\textbf{s})
\end{gather*}

\subsubsubsection{Well-formed declarations}
We use an inference of the form
\begin{gather*}
    syms \vdash WFDeclaration(\textbf{d})
\end{gather*}
to state that, in context $syms$, the declaratio \textbf{d} is well
formed, and
\begin{gather*}
    entry(syms,\textbf{d}) = ent
\end{gather*}
to state that its corresponding symbol table entry is $ent$.

\subsubsubsection{Well-formed constant expressions}
To determine the value of an expression, we use an inference of the form
\begin{gather*}
    syms \vdash \textbf{c} \xrightarrow{e} v
\end{gather*}
to state that the constant expression \textbf{c} evaluates to the value $v$ in the context 
of $syms$.

\subsubsubsection{Well-formed blocks}
The notation $syms \oplus \{ \textbf{id} \in X \bullet (\textbf{id} \mapsto ent)\}$ 
stands for the symbol table $syms$ but with the entried for all the indentifiers in the set 
X added (or replaced) by teh vlaue of the expression $ent$.

\subsubsubsection{Examples}
Rule 3.1 Integer value
\begin{gather*}
    syms \vdash \textbf{n} : int
\end{gather*}
Rule 3.2 Symbolic constant
\begin{gather*}
    \textbf{id} \in \textbf{dom}(syms)\\
    syms(\textbf{id}) = ConstEntry(T, v)\\
    \hline
    syms \vdash \textbf{id} : T
\end{gather*}
Rule 3.3 Variable identifier
\begin{gather*}
    \textbf{id} \in \textbf{dom}(syms)\\
    syms(\textbf{id}) = VarEntry(T, v)\\
    \hline
    syms \vdash \textbf{id} : T
\end{gather*}
Rule 3.4 Unary negation
\begin{gather*}
    syms \vdash \textbf{e} : int\\
    \hline
    syms \vdash \textbf{op($- \_$,e)} : int
\end{gather*}
Rule 3.5 Binary operator
\begin{gather*}
    syms \vdash \textbf{e$_1$} : T_1\\
    syms \vdash \textbf{e$_2$} : T_2\\
    syms \vdash \_ \odot \_ : T_1 \times T_2 \rightarrow T_3\\
    \hline
    syms \vdash \textbf{op($\_ \odot \_ $,(e$_1$,e$_2$))} : T_3
\end{gather*}
Rule 3.6 Dereference
\begin{gather*}
    syms \vdash \textbf{e} : ref(T)\\
    \hline
    syms \vdash \textbf{e} : T
\end{gather*}
Rule 3.7 Widen subrange
\begin{gather*}
    syms \vdash \textbf{e} : subrange(T, i, j)\\
    \hline
    syms \vdash \textbf{e} : T
\end{gather*}
Rule 3.8 Narrow subrange
\begin{gather*}
    syms \vdash \textbf{e} : T\\
    i <= j\\
    T \in \{int,bolean\}\\
    \hline
    syms \vdash \textbf{e} : subrange(T, i, j)
\end{gather*}