\subsection{Static Semantics}

\subsubsection{Types}
\subsubsubsection{Function Type}
$T_1 \rightarrow T_2$ indicates a mapping from an argument type $T_1$ to a result type $T_2$.

\subsubsubsection{Product Type}
$T_1 \times T_2$ is the type consisting of all pairs with first element of type $T_1$ and 
second element of type $T_2$.

\subsubsection{Well-typed Expressions}
\subsubsubsection{Type Inferences}
To state that the expression \textbf{e} is well typed in the context of the symbol table and has 
type $T$, we write $syms \vdash \textbf{e} : T$.

\subsubsubsection{Type Coercion}
We can write rules which allow us to implicitly coerce types from $T_1$ to $T_2$ if required. 
E.g. the dereference rule (3.6).

\subsubsubsection{Well-formed statements}
To state a statement is well formed, we write
\begin{gather*}
    syms \vdash WFStatement(\textbf{s})
\end{gather*}

\subsubsubsection{Well-formed declarations}
We use an inference of the form
\begin{gather*}
    syms \vdash WFDeclaration(\textbf{d})
\end{gather*}
to state that the declaration \textbf{d} is well
formed, and
\begin{gather*}
    entry(syms,\textbf{d}) = ent
\end{gather*}
to state its corresponding syms entry is $ent$.

\subsubsubsection{Well-formed constant expressions}
We use an inference of the form
\begin{gather*}
    syms \vdash \textbf{c} \xrightarrow{e} v
\end{gather*}
to state that the constant expression \textbf{c} evaluates to the value $v$.

\subsubsubsection{Well-formed blocks}
The notation $syms \oplus \{ \textbf{id} \in X \bullet (\textbf{id} \mapsto ent)\}$ 
stands for the symbol table $syms$ but with the entried for all the indentifiers in the set 
X added (or replaced) by teh vlaue of the expression $ent$.

\subsubsubsection{Examples}
Rule 3.1 Integer value
\begin{gather*}
    syms \vdash \textbf{n} : int
\end{gather*}
Rule 3.2 Symbolic constant
\begin{gather*}
    \textbf{id} \in \textbf{dom}(syms)\\
    syms(\textbf{id}) = ConstEntry(T, v)\\
    \hline
    syms \vdash \textbf{id} : T
\end{gather*}
Rule 3.3 Variable identifier
\begin{gather*}
    \textbf{id} \in \textbf{dom}(syms)\\
    syms(\textbf{id}) = VarEntry(T, v)\\
    \hline
    syms \vdash \textbf{id} : T
\end{gather*}
Rule 3.4 Unary negation
\begin{gather*}
    syms \vdash \textbf{e} : int\\
    \hline
    syms \vdash \textbf{op($- \_$,e)} : int
\end{gather*}
Rule 3.5 Binary operator
\begin{gather*}
    syms \vdash \textbf{e$_1$} : T_1\\
    syms \vdash \textbf{e$_2$} : T_2\\
    syms \vdash \_ \odot \_ : T_1 \times T_2 \rightarrow T_3\\
    \hline
    syms \vdash \textbf{op($\_ \odot \_ $,(e$_1$,e$_2$))} : T_3
\end{gather*}
Rule 3.6 Dereference
\begin{gather*}
    syms \vdash \textbf{e} : ref(T)\\
    \hline
    syms \vdash \textbf{e} : T
\end{gather*}
Rule 3.7 Widen subrange
\begin{gather*}
    syms \vdash \textbf{e} : subrange(T, i, j)\\
    \hline
    syms \vdash \textbf{e} : T
\end{gather*}
Rule 3.8 Narrow subrange
\begin{gather*}
    syms \vdash \textbf{e} : T\\
    i <= j\\
    T \in \{int,bolean\}\\
    \hline
    syms \vdash \textbf{e} : subrange(T, i, j)
\end{gather*}