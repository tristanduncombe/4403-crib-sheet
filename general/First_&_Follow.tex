\subsection{First \& Follow Sets}

\subsubsection{First Sets}
The first set for a construct $\alpha$ records
\begin{itemize}
    \item the set of terminal symbols $\alpha$ can start with, and
    \item if $\alpha$ is nullable, it contains the empty string $\epsilon$ to indicate that.
\end{itemize}

\subsubsubsection{Calculating First Sets}
Let
\begin{itemize}
    \item $\alpha$ be a terminal symbok,
    \item $\alpha_1,\alpha_2,\dots,\alpha_n$ be strings of (terminal and nonterminal) symbols, and
    \item $A$ be a nonterminal symbol defined by a single production
\end{itemize}
\begin{align*}
    A \rightarrow \alpha_1 | \alpha_2 | \dots | \alpha_n,
\end{align*}
then
\begin{align*}
    Fst(\epsilon) &= \{\epsilon\}\\
    Fst(\alpha) &= \{\alpha\}\\
    Fst(\alpha_1 | \alpha_2 | \dots | \alpha_n) &= Fst(\alpha_1) \cup Fst(\alpha_2)\\
                                                  &\ \ \ \ \cup \dots \cup Fst(\alpha_n)\\  
    Fst(A) &= Fst(\alpha_1 | \alpha_2 | \dots | \alpha_n)\\
\end{align*}

Let $S_1,S_2,\dots,S_n$ be (terminal or nonterminal) symbols, then
\begin{align*}
    F&st(S_1 S_2 \dots S_n) =\\
    &Fst(S_1) - \{\epsilon\}\\
    &\cup Fst(S_2) - \{\epsilon\} &&\text{if $S_1$ is nullable}\\
    &\cup \dots\\
    &\cup Fst(S_i) - \{\epsilon\} &&\text{if $S_1$-$S_{i-1}$ is nullable}\\
    &\cup \dots\\
    &\cup Fst(S_n) - \{\epsilon\} &&\text{if $S_1$-$S_{n-1}$ is nullable}\\
    &\cup \{\epsilon\} &&\text{if $S_1$-$S_n$ is nullbale}
\end{align*}

\subsubsubsection{Calculating Algorithmically}
Start with the first sets for all nonterminals being the empty set and note that the fisrt set for every terminal symbol, $\alpha$, is the singleton set $\{\alpha\}$.
We then make a pass over all production in a grammer considering all atlternatives and process as follows.
\begin{itemize}
    \item If there is a production of the form $N \leftarrow \epsilon$, we add $\epsilon$ to the first set of $N$.
    \item If there is a production of the form $N \leftarrow S_1 S_2 \dots S_n$, then for each $i \in 1..n$, if for all $j \in 1..i-1$, $S_j$ is nullabe, we add the current first set for $S_i$ minus $\epsilon$ to the first set for $N$.
    \item If every construct $S_1,...,S_n$ is nullabe, we add $\epsilon$ to the first set for $N$. 
\end{itemize}
We repeat the the passes until no set is modified in the pass, in which case we are finished.
\subsubsubsubsection{Example}
\begin{align}
    A &\leftarrow B\,x | C\\
    B &\leftarrow C\,y | D\\
    C &\leftarrow D\,z | \epsilon\\
    D &\leftarrow A\,w
\end{align}
\begin{center}
\begin{tabular}{|c|c|c|c|c|c|}
    \hline
    $A$ & $\{\}$ & $\{\epsilon\}$ & $\{\epsilon,y\}$ &  $\{\epsilon,y,w\}$ &  $\{\epsilon,y,w\}$\\
    \hline
    $B$ & $\{\}$ & $\{y\}$ & $\{y\}$ &  $\{y,w\}$ &  $\{y,w\}$\\
    \hline
    $C$ & $\{\epsilon\}$ & $\{\epsilon\}$ & $\{\epsilon,w\}$ &  $\{\epsilon,w,y\}$ &  $\{\epsilon,w,y\}$\\
    \hline
    $D$ & $\{\}$ & $\{w\}$ & $\{w,y\}$ &  $\{w,y\}$ &  $\{w,y\}$\\
    \hline
\end{tabular}
\end{center}



\subsubsection{Follow Sets}
The follow set for a nonterminal, $N$, is the set of terminal symbols that may follow $N$ in any context within the grammar.
End-of-file is represented by the special terminal symbol $\$$ in which $a$ follows $N$.

A nonterminal, $N$, is followed by a terminal symbol, $a$, if there is a derivation from $S\$$

\subsubsubsection{Calculating Follow Sets}
We compute the Follow set for a nonterminal, $N$, using two rules.
\begin{itemize}
    \item If there is a production of the form
    \begin{align*}
        A \rightarrow \alpha N \beta
    \end{align*}
    then any symbols that can start $\beta$ can follow $N$, and hence $Follow(N)$ must include all the terminal symbols in $First(\beta)$. Note that $\epsilon$ is not included even if it appears in $First(\beta)$.
    \begin{align*}
        First(\beta) - \{\epsilon\} \subseteq Follow(N)
    \end{align*}
    \item If there is a production of the form
    \begin{align*}
        A \rightarrow \alpha N \beta
    \end{align*}
    and $\beta$ is nullable, then any token that can follow $A$ can also follow $N$. Hence,
    \begin{align*}
        Follow(A) \subseteq Follow(N)
    \end{align*}
    The case where $\beta$ is nullable includes the case when $\beta$ is empty and the production is of the form
    \begin{align*}
        A \rightarrow \alpha N
    \end{align*}
\end{itemize}

\subsubsubsection{Calculating Algorithmically}
Start with all nonterminal symbols having an empty follow set, $\{\}$, except for the start symbol, $S$, which has the follow set $\{\$\}$.

We make a pass through the grammar examining the right side of every production. 
For each occurence of a nonterminal within the right side of some production, 
we augment the Follow set for that nonterminal according to the following process. 
Assume we are processing an occurence of $N$ and the production is of the form
\begin{align*}
    A \rightarrow \alpha N \beta
\end{align*}
we add $First(\beta) - \{\epsilon\}$ to the Follow set computer for $N$ so far, 
and if $\beta$ is nullable, we also add the current Follow set for $A$ to the Follow set for $N$.

After making a complete pass, we repeat the process with the Follow sets computed so far until no Follow sets are modified, in which case we are done.
\subsubsubsubsection{Example}
\begin{equation*}
    \begin{aligned}[c]
        S &\rightarrow x A B\\
        A &\rightarrow y | z B\\
        B &\rightarrow \epsilon | A x
    \end{aligned}
    \qquad 
    \begin{aligned}[c]
        Fst(S) &= \{x\}\\
        Fst(A) &= \{y, z\}\\
        Fst(B) &= \{\epsilon, y, z\}
    \end{aligned}
\end{equation*}

\begin{center}
    \begin{tabular}{|c|c|c|c|c|}
        \hline
        $S$ & $\{\$\}$ & $\{\$\}$ & $\{\$\}$ & $\{\$\}$\\
        \hline
        $A$ & $\{\}$ & $\{y,z,\$,x\}$ & $\{y,z,\$,x\}$ & $\{y,z,\$,x\}$\\
        \hline
        $B$ & $\{\}$ & $\{\$,y,z\}$ & $\{\$,y,z,x\}$ & $\{\$,y,z,x\}$\\
        \hline
    \end{tabular}
\end{center}

\subsubsection{LL(1) Grammar}
A BNF grammer is LL(1) if for each nonterminal, $N$, wher $N \rightarrow \alpha_1 | \alpha_2 | \dots | \alpha_n$,
\begin{itemize}
    \item the First sets for each pair of alternatives for $N$ are disjoint, and
    \item if $N$ is nullable, $First(N)$ and $Follow(N)$ are disjoin.
\end{itemize}
