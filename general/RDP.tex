\subsection{Recusive-Descent Parsing}
\subsubsection{Backus-Naur Form (BNF)}
Allows a context-free grammar to be described by a set of productions of the form
$N \rightarrow \alpha$
Where $N$ is a nonterminal symbol and $\alpha$ is a (possibly empty) sequence of terminal and nonterminal symbols

The set of productions:

$N \rightarrow \alpha_1$ \\
$N \rightarrow \alpha_2$ \\
$\ldots$ \\[0.1cm]
$N \rightarrow \alpha_n$

can be writen as
$N \rightarrow \alpha_1 | \alpha_2 | \ldots | \alpha_n$

\subsubsection{Extended BNF (EBNF)}
Extends BNF to allow:

An \emph{optional} syntatic construct [in square brackets]

$RelCondition \rightarrow Exp \ [RelOp \ Exp]$ \newline

A \emph{reptition} construct {in curly braces}\newline

$StatementList \rightarrow Statement \ \{SEMICOLON \ Statement \}$ \newline


The extended notation does not add any expressive power - any EBNF grammar can be converted to a BNF grammar\dots

\subsubsection{Translating EBNF to BNF}

Replace optional construct $[S]$ by a new nonterminal $OptS$:

$OptS \rightarrow S | \epsilon$ \newline
For example, we can rewrite
$RelCondition \rightarrow Exp \ [RelOp \ Exp]$ \newline
as

$RelCondition \rightarrow Exp \ OptRelExp$
$OptRelExp \rightarrow RelOp \ Exp | \epsilon$

Replace grouping construct ( $S$ ) by a new nonterminal $GrpS$:

$GrpS \rightarrow S$

For example, we can rewrite

$RepF \rightarrow (TIMES \ | \ DIVIDE) \ Factor \ RepF \ | \epsilon$

as 

$RepF \rightarrow TermOp \ Factor \ RepF \ | \epsilon$ \\
$TermOp \rightarrow TIMES \ | \ DIVIDE$

\subsubsection{Recursive-descent parsing}
A recursive-descent parser is a recursive program to recognise sentences in a language:
\begin{itemize}
    \item Input is a stream of lexical tokens, represented by tokens of type TokenStream. 
    \item Each nonterminal symbol $N$ has a method $parseN$ that:
    \begin{itemize}
        \item recognises the longest string of lexcial tokens in the input stream, starting from the current token, derivable from $N$;
        \item as it parses the input it moves the current location forward;
        \item when it has finished parsing N, the current token is the token immediately following the last token matched as part of $N$ (which may be at the end-of-file token).
    \end{itemize}
\end{itemize}

\subsubsection{matching a list of alternatives: example}
\begin{scriptsize}
$Factor \rightarrow \ $ \\ 
$LPAREN \ RelCondition \ RPAREN \ |$ \\
$\ NUMBER \ |$ \\
$\ LValue $
\begin{lstlisting}[language=Java]
void parseFactor (){
    if(tokens.isMatch(Token.LPAREN)){
        tokens.match(Token.LPAREN);
        parseRelCondition();
        tokens.match(Token.RPAREN);
    } else if(tokens.isMatch(Token.NUMBER)){
        tokens.match(Token.NUMBER);
    } else if
        (tokens.isMatch(Token.IDENTIFIER)){
        parseLValue();
    } else {
        errors.error ("Syntax error");
    }
}
\end{lstlisting}
\end{scriptsize}

\subsubsection{matching an optional construct: example}
\begin{scriptsize}
$RelCondition \rightarrow \ Exp \ [ \ RelOp \ Exp \ ]$
\begin{lstlisting}[language=Java]
void parseRelCondition(){
    parseExp();
    if(tokens.isIn(REL_OPS_SET)){
        parseRelOp();
        parseExp();
    }
}
\end{lstlisting}
\end{scriptsize}

\subsubsection{matching a repetition construct: example}
\begin{scriptsize}
$Term \rightarrow \ Factor \ { \ ( \ TIMES \ | \ DIVIDE \ ) \ Factor \ }$
\begin{lstlisting}[language=Java]
void parseTerm(){
    parseFactor();
    while(tokens.isIn(TERM_OPS_SET)){
        if(tokens.isMatch(Token.TIMES)){
            tokens.match(Token.TIMES);
        } else if 
            (tokens.isMatch(Token.DIVIDE)){
            tokens.match(Token.DIVIDE);
        } else {
            fatal("Internal Error");
            // unreachable
        }
        parseFactor();
    }
}
\end{lstlisting}
\end{scriptsize}

\subsubsection{recovery strategy for matching a single token}
\begin{scriptsize}
\begin{lstlisting}[language=Java, emph={STATEMENT_START_SET}, emphstyle=\textbf]
void parseWhileStatement(){
    tokens.match(Token.KW _WHILE);
    parseCondition();
    tokens.match(Token.KW _DO, 
        STATEMENT_START_SET );
    parseStatement();
}
\end{lstlisting}
\end{scriptsize}

Generally, it is  
\begin{lstlisting}[language=Java]
tokens.match(expected, follows)
\end{lstlisting}

maybe add parsing recovery sets here?

\subsubsection{LL(1) grammars}
We indicated above that there are some restrictions on gram-
mars to ensure they are suitable for recursive-descent predic-
tive parsing. The class of grammars that are suitable is re-
ferred to as LL(1), where the first “L” refers to the fact that
the parsing of the input is from Left to right, the second “L”
refers to the fact that they produce a Leftmost derivation se-
quence, and the “1” indicates that their parsers use one symbol
lookahead (i.e., the current token is the lookahead).

Definition 1 (LL(1) Grammar) A BNF grammar is LL(1) if
for each nonterminal, N, the first sets for each pair of alter-
native productions for N are disjoint, and if N is nullable,
First(N) and Follow(N) are disjoint