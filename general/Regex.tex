\subsection{Regular Expressions}
The syntax of regular expressions in BNF is as follows:
\begin{align*}
    e :: = a | \epsilon | \theta | e"|"e | e \ e | e^{"*"} | (e)
\end{align*}
Note that regex uses "|" to list alternatives similar to BNF notation.

\subsubsubsection{Language of a regular expression}
Given an alphabet $\Sigma$, the set of regular expressions defines a language (i.e., a set opf strings of symbols from the alphabet). Let $"a"$ be a symbol, and $e$ and $f$ regular expressions.

\begin{itemize}
    \item $\pazocal{L}(a) = \{"a"\}$
    \item $\pazocal{L}(\epsilon) = \{"" \}$
    \item $\pazocal{L}(\theta) = \{ \}$
    \item $\pazocal{L}(e | f) = \pazocal{L}(e) \cup \pazocal{L}(f)$
    \item $\pazocal{L}(e \ f) = \pazocal{L}(e) \smallfrown \pazocal{L}(f)$
    \item $\pazocal{L}(e^{*}) = \cup _{n \in \mathbb{N}} (\pazocal{L}(e))^n$
    \item $\pazocal{L}((e)) = \pazocal{L}(e)$
\end{itemize}

\subsubsubsection{Concatenation of languages}
The concatenation of two langauges (sets of string), $L_1$ and $L_2$ is defined as the set of all strings formed by taking a string, $s_1$ from $L_1$ and a string, $s_2$ from $L_2$, and concatenating them. For example, 
\begin{align*}
    \{a, b\} \smallfrown \{c, d\} = \{ac, ad, bc, bd\}
\end{align*}

\subsubsubsection{Iteration of a language}
The iteration, $L_n$, of a language, $L$, to the power $n$, consists of $L$ concatenated with itself $n$ times. 
\begin{align*}
    L^0 = \{\} \\
    L^1 = L \\
    L^2 = L \smallfrown L \\
    L^3 = L \smallfrown L \smallfrown L \\
    \dots
\end{align*}
$\pazocal{L}(a|b) = {a, b}$  and
\begin{align*}
    \{a, b\}^0 = \{\}\\
    \{a, b\}^1 = \{a, b\}\\
    \{a, b\}^2 = \{aa, ab, ba, bb\}\\
    \dots
\end{align*}

Hence $\pazocal{L}((a|b)^*) = \{, a, b, aa, ab, ba, bb \dots \}$ \\
Note that this does not include any infinite strings of symbols because the iteration only generates finite concatenations.
\subsubsubsection{Finite Automata}
\begin{itemize}
    \item A finite automaton is a finite state machine consisting of a fine set of states with labelled transitions between states. 
    \item Each transitioin is labelled with either a symbol ($a \in \Sigma$) or empty ($\epsilon$).
    \item We distinguish between deterministic and nondeterminisitc finite automata.
    \item For a deterministic finite automaton (DFA) empty transitions are not allowed, and for each state and symbol the next statte, if there is one, is uniquely determined. 
    \item For a nondeterministic finite automaton (NFA) empty transitions are allowed, and there may be multiple transitions from a state to a different next states on the same input symbol.
\end{itemize}

\subsubsubsection{Deterministic Finite Automaton (DFA)}
A DFA, $D$, consists of
\begin{itemize}
    \item A finite alphabet of symbols, $\Sigma$
    \item A finite set of states, $S$
    \item A transition function, $T : \Sigma \times S \rightarrow S$, which maps an (input) symbol and a (current) state to the (next) state; the function $T$ may not be defined for all pairs of symbols and state. 
    \item A start state, $s_0$ 
    \item A set of final (or accepting) states, $F$
\end{itemize}

\image{DFA.png}
\begin{itemize}
    \item $\Sigma = {a,b,c}$
    \item $S = \{1,2,3,4\}$
    \item $T(a, 1) = 2 $
    \item $T(c, 1) = 4$
    \item $T(b, 2) = 3$
    \item $s_0 = 1$
    \item $F = \{3,4\}$
\end{itemize}

\subsubsubsection{Language of a DFA}
The language $\pazocal{L}(D)$, of a DFA $D$, is the set of finite strings fo symbols from $\Sigma$ such that $c \in \pazocal{L}(D)$ if and only if there exists a sequence of states $s_1, s_2, \dots, s_n$ where $n$ is the legnth of $c$, such that
\begin{align*}
    T(c_1, s_0) = s_1 \\
    T(c_2, s_1) = s_2 \\
    \dots \\
    T(c_n, s_{n-1}) = s_n
\end{align*}
and $s_0$ is the start state and $s_n$ is in the set of final (or accepting) states F.

The language of the above DFA is $\pazocal{L}(D) = \{ab, c\}$

\subsubsubsection{Nondeterministic Finite Automaton (DFA)}
A NFA, $N$, is different to a DFA in that its 

Transition function is $T : (\Sigma \cup \{ \epsilon \}) \times S \rightarrow \mathbb{P} S$, which maps an (input) symbol and a (current) state to a  set of possible (next states); the function $T$ may not be defined for all pairs of symbol and state.


\subsubsubsection{NFA}


\begin{itemize}
    \item $\Sigma = {a,b,c}$
    \item $S = {1, 2, 3, 4, 5, 6, 7}$
    \item $T(\epsilon) = \{2,6\}    T(a,2) = \{3\}$
    \item $T(b,3) = \{4\}    T(\epsilon,4) = \{5\}$
    \item $T(c,6) = \{7\}    T(\epsilon,7) = \{5\}$
    \item $s_0 = 1$
    \item $F = \{5\}$
\end{itemize}
\image{NFA.png}

\subsubsubsection{Language of a NFA}
The language $\pazocal{L}(N)$, of a NFA $N$, is the set of finite strings fo symbols from $\Sigma$ such that $c \in \pazocal{L}(N)$ if and only if there exists a sequence $c' \in (\Sigma \cup \{\epsilon \})^*$, such that removing all the $\epsilon$ elements from $c'$ gives $c$ and there exists a sequence of states $s_1, s_2, \dots, s_n$ where $n$ is the length of $c'$, such that
\begin{align*}
    T(c'_1, s_0) = s_1 \\
    T(c'_2, s_1) = s_2 \\
    \dots \\
    T(c'_n, s_{n-1}) = s_n
\end{align*}
and $s_0$ is the start state and $s_n$ is in the set of final (or accepting) states F.

The language of the above NFA is $\pazocal{L}(D) = \{ab, c\}$ Empty transitions are used to generate the strings by then omitted from the strings. 

\subsubsubsection{From a Regex to an NFA}
\begin{itemize}
    \item The translation of a regular expression to a nondeterministic finite automaton is based on the structure of the regular expression.
    \item For each of the syntactic forms of regular expressions given in Definition 1, an NFA is given for that form.
    \item If the syntactic form contains sub-expressions (e.g. e | f contains sub-expressions e and f ) the NFA for that form is constructed from the NFAs for the sub-expressions e and f .
\end{itemize}

\image{accepting_state.png}
The final (acceptiong states are indicted by double circle.)

\image{concatenation.png}
The final state of the NFA for $e$ is no longer a final in the NFA for $e$ $f$.

\subsubsubsection{Example a b $|$ c}
\image{abc_example.png}

\subsubsubsection{Alternation e $|$ f}
The NFAs for e and f are the same as for concatenation.
\image{alternationef.png}
The final states of the NFAs for both e and f are no longer final
states in the NFA for e $|$ f.

\subsubsubsection{Repetition e*}
\image{nfa_e.png}
The final state in the NFA for $e$ is no longer an final state in the NFA for $e^*$.

\subsubsubsection{NFA for a | b}
\image{nfa_ab.png}

\subsubsubsection{NFA for (a | b)*}
\image{nfa_eps.png}

\subsubsubsection{From an Regex to a NFA}
In the translation from a regular expression, $r$, to an NFA, thegenerated NFA has a few properties that do not necessarily hold for an arbitrary NFA (i.e. one not generated from a regular expression).
\begin{itemize}
    \item The NFA has a single final (accepting) state.
    \item The initial state of the NFA only has outgoing transitions.
    \item The final state only has incoming transitions.
\end{itemize}
The translation rules preserve these properties.

\subsubsubsection{From NFA to DFA}
A DFA cannot have 
\begin{itemize}
    \item more than one transition leaving a state on the same symbol
    \item any empty transitions
\end{itemize}
An NFA N can be translated to an equivalent DFA D.
\begin{itemize}
    \item equivalent in the sense that they accept the same languages, i.e., $\pazocal{L}(N) = \pazocal{L}(D)$.
\end{itemize}

\subsubsubsection{From NFA to DFA}
An NFA is transformed to a DFA in which the labels on the states of the DFA are sets of states from the NFA.
The sets of states that label a DFA state are formed by collecting all the states that can be reached from NFA states by empty transitions. 

\subsubsubsection{Empty Closure of a state}
Empty Closure of a state
The empty closure of a state $x$ in an NFA $N$, $\epsilon$-closure$(x, N)$, is
the set of states in $N$ that are reachable from x via any number
of empty transitions

\subsubsubsection{Empty Closure of a set of states}
The empty closure of a set of states $X$ in an NFA $N$, $\epsilon$-closure$(X , N)$, is the set of states in $N$ that are reachable from any of the states in $X$ via any number of empty transitions. 

\begin{align*}
    \epsilon \text{-closure}(X, N) = \underset{x \in X}{\cup} \ \epsilon \text{-closure}(x, N)
\end{align*}

Empty close of $a \ b \ | \ c$

\begin{tabular}{|c|c|}
    \hline
    State $x$ & $\epsilon$-closure$(x, \text{NFA})$ \\
    \hline
    1 & \{1, 2, 6\} \\
    2 & \{2\} \\
    3 & \{3\} \\
    4 & \{4, 5\} \\
    5 & \{5\} \\
    6 & \{6\} \\
    7 & \{7, 5\} \\
    \hline
\end{tabular}

\subsubsubsection{Constructing the DFA from the NFA}

The label of the start state of the DFA consists of the set of states containing the start state $s_0$ of the NFA plus all the states in the NFA that are reachable from its start state by one or more empty transitions. 

\begin{align*}
    S_0 = \epsilon\text{-closure}(s_0, N). 
\end{align*}

The process for forming a DFA works with a set of unmarked DFA states by selecting an unmarked DFA state and considering all transitions from it on symbols. The initial set of unmarked states contains just $S_0$.

\subsubsection{From NFA to DFA}

The following process is repeated until there are no unmarked DFA states left:
\begin{itemize}
    \item An unmarked DFA state $S$ is selected (the first one is $S_0$).
    \item For each symbol $a$,
    \begin{itemize}
        \item we consider the set of states that can be reached from any state in $S$ by a transition on $a$; call this set of states $X$.
        \item If $X$ is nonempty, we add a new state to the DFA labeled with $X' = \epsilon\text{-closure}(X, N)$, unless a state with that label already exists, in which case it is reused.
        \item A transition from $S$ to $X'$ on $a$ is added to the DFA.
    \end{itemize}
    \item The state $S$ is marked as having been processed.
\end{itemize}

\subsubsubsection{From NFA to DFA for a b | c}

\image{nfatodfa.png}

\subsubsubsection{From NFA to DFA for (b | c)* a*}

\image{nfatodfabca.png}

\subsubsubsection{Minimising a DFA}

The DFA generated in the above example can be simplified to one with only two states. Below is a simplified version of the DFA with the states relabelled to to A, B, C and D.
\image{minimising_dfa.png}

To minimise the DFA we merge states that have the same
transition to the equivalent states. For the example, A, B and C
are equivalent.

\image{minimised.png}

Start by partitioning states into a group of final states and a group of non-final states. For the example, all states are all final so just get the one group \{A, B, C, D\}.
For each symbol $x$, we require all states in a group $G1$ to transition to the same group $G2$. For the example, a transition on $b$ takes states A, B and C back to the same group \{A, B, C\} but there is no transition on $b$ from state D and hence we must split the group into groups \{A, B, C\} and \{D\}.
Once that is done for every symbol $x$, the transitions from all states in each group on $x$ are to the same group, and hence we have finished.

\subsubsubsection{Regular expressions for lexical analysis}

\begin{itemize}
    \item The lexical analyser generator JFlex makes use of translating regular expressions into DFAs in order to build a scanner.
    \item The input to the lexical analyser generator is a list of regular expressions along with an action to be taken when that token is matched.
    \item The generated lexical analyser matches
    \begin{itemize}
        \item the longest prefix of the input from the current position that matches any of the regular expressions in the list and
        \item if the prefix of the input matches more than one regular expression, the action associated with the first matching regular expression in the list is executed.
    \end{itemize}
\end{itemize}


\begin{itemize}
    \item For PL0 the string "end" followed by a blank matches both
    \begin{itemize}
        \item the regular expression for the keyword "end" and
        \item the regular expression for an identifier.
    \end{itemize}
    \item The action selected for "end" is that for the keyword "end" because the regular expression for the keyword appears before the regular expression for an identifier.
    \item The string "ending" followed by a blank matches only the regular expression for an identifier; it will not be split into the keyword "end" followed by the identifier "ing".
    \item But the string "end ing " will be matched as the keyword "end" followed by the identifier "ing".
\end{itemize}