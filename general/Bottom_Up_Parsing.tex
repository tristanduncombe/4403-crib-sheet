\subsection{Bottom Up Parsing}

\subsubsection{Shift/Reduce Parsing}
Makes use of a parse stack and has three actions:
\begin{itemize}
    \item \textbf{shift} - push the next input symbol onto the stack,
    \item \textbf{reduce} - if a sequence of symbols on the top of the stack, $\alpha$, 
    matches the right side of some production $N \rightarrow \alpha$, then the sequence 
    $\alpha$ on top of the stack is replaced by $N$.
    \item \textbf{accept} - if the stack contains just the start symbol and there is no
    input left, the input has been recognised and is accepted.
\end{itemize}

\subsubsubsection{LR(0) Grammars}
An LR(0) parsing item is of the form
\begin{align*}
    N \rightarrow \alpha \bullet \beta
\end{align*}
which indicates that, in matching $N$, $\alpha$ has been matched and $\beta$ is yet to 
be matched where
\begin{itemize}
    \item $N$ is a nonterminal symbol,
    \item $\alpha$ and $\beta$ are possibly empty sequences of (terminal and nonterminal) 
    symbols such that $N \rightarrow \alpha\ \beta$ is a production of the grammer, and
    \item $\bullet$ marks the current position in matching the right side.
\end{itemize}

\subsubsubsubsection{Parsing Automaton}
An LR(0) parsing automaton consists of
\begin{itemize}
    \item a finite set of states, each of which consists of a set of LR(0) parsing items, and
    \item transitions between states, each of which is labelled by a transition symbol.
\end{itemize}
Each state in an LR(0) parsing automaton must have only one associated parsing action.