\subsection{Bottom Up Parsing}

\subsubsection{Shift/Reduce Parsing}
\subsubsubsubsection{Example Table}
\begin{equation*}
    \begin{aligned}[c]
        S \rightarrow A
    \end{aligned}
    \qquad 
    \begin{aligned}[c]
        A \rightarrow (\ A\ )
    \end{aligned}
    \qquad 
    \begin{aligned}[c]
        A \rightarrow a
    \end{aligned}
\end{equation*}
\begin{center}
    \begin{tabular}{|lr|l|}
        \hline
        Parsing stack \hspace{0.5cm} & Input & Parsing action \\
        \hline
        \$0 & ((a))\$ & shift \\
        \$0(2 & (a))\$ & shift \\
        \$0(2(2 & a))\$ & shift \\
        \$0(2(2a3 & ))\$ & reduce $A \rightarrow a$ \\
        \$0(2(2A4 & ))\$ & shift \\
        \$0(2(2A4)5 & )\$ & reduce \( A \rightarrow (A)\) \\
        \$0(2A4 & )\$ & shift \\
        \$0(2A4)5 & \$ & reduce \( A \rightarrow (A)\) \\
        \$0A1 & \$ & accept \\
        \hline
    \end{tabular}
\end{center}


\subsubsubsubsection{Parsing Conflicts}
If a state in an LR(0) parsing automaton has more than one action, 
there is a parsing action conflict.

A grammar is LR(0) if none of the states in its parsing 
automaton contains a conflict.

\subsubsubsubsection{Derived Items}
If a state has an LR(1) item of the form
\begin{align*}
    [N \rightarrow \alpha \bullet M \beta, T]
\end{align*}
where the nonterminal M has productions
\begin{align*}
    M \rightarrow \alpha_1 | \alpha_2 | \dots | \alpha_m
\end{align*}
\& $T = \{a_1, a_2, \dots, a_n\}$, then the state also includes the derived items
\begin{align*}
    [N \rightarrow \bullet \alpha_1, T]
    \dots
    [N \rightarrow \bullet \alpha_m, T]
\end{align*}
where if $\beta$ is not nullable $T' = First(\beta)$ \& if $\beta$ is nullable, 
$T' = First(\beta) - \{\epsilon\} \cup T$.

\subsubsubsubsection{Parsing Action Conflicts}
If a state in an LR(1) parsing automaton has more than one action for a look-ahead 
terminal symbol, there is a parsing action conflict.

A grammar is LR(1) if none of the states in its parsing automaton contains a conflict.