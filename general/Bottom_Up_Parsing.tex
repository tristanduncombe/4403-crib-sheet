\subsection{Bottom Up Parsing}

\subsubsection{Shift/Reduce Parsing}
Makes use of a parse stack and has three actions:
\begin{itemize}
    \item \textbf{shift} - push the next input symbol onto the stack,
    \item \textbf{reduce} - if a sequence of symbols on the top of the stack, $\alpha$, 
    matches the right side of some production $N \rightarrow \alpha$, then the sequence 
    $\alpha$ on top of the stack is replaced by $N$.
    \item \textbf{accept} - if the stack contains just the start symbol and there is no
    input left, the input has been recognised and is accepted.
\end{itemize}

\subsubsubsection{LR(0) Grammars}
An LR(0) parsing item is of the form
\begin{align*}
    N \rightarrow \alpha \bullet \beta
\end{align*}
which indicates that, in matching $N$, $\alpha$ has been matched and $\beta$ is yet to 
be matched where
\begin{itemize}
    \item $N$ is a nonterminal symbol,
    \item $\alpha$ and $\beta$ are possibly empty sequences of (terminal and nonterminal) 
    symbols such that $N \rightarrow \alpha\ \beta$ is a production of the grammer, and
    \item $\bullet$ marks the current position in matching the right side.
\end{itemize}

\subsubsubsubsection{Parsing Automaton}
An LR(0) parsing automaton consists of
\begin{itemize}
    \item a finite set of states, each of which consists of a set of LR(0) parsing items, and
    \item transitions between states, each of which is labelled by a transition symbol.
\end{itemize}
Each state in an LR(0) parsing automaton must have only one associated parsing action.

\subsubsubsubsection{Automaton States}
If a state has an LR(0) item of the form
\begin{align*}
    N &\rightarrow \alpha \bullet M \beta \\
\end{align*}
where the nonterminal $M$ has productions
\begin{align*}
    M &\rightarrow \alpha_1 \\
    M &\rightarrow \alpha_2 \\
    &\vdots \\
    M &\rightarrow \alpha_m \\
\end{align*}
then the state also includes the derived items
\begin{align*}
    M &\rightarrow \bullet \alpha_1 \\
    M &\rightarrow \bullet \alpha_2 \\
    &\vdots \\
    M &\rightarrow \bullet \alpha_m
\end{align*}

\subsubsubsubsection{Goto States}
If a state $s_i$ has an LR(0) item of the form
\begin{align*}
    N \rightarrow \alpha \bullet x \beta
\end{align*}
where $x$ is either a terminal symbol or a nonterminal symbol, then there is a goto state, 
$s_j$, from the state $s_1$ on $x$, and $s_j$ includes a kernel item of the form
\begin{align*}
    N \rightarrow \alpha x \bullet \beta
\end{align*}
If there are multiple items in $s_i$ with the same $x$ immediately to the right of the 
$\bullet$ then the goto state $s_j$ includes all those items but with the $\bullet$ after 
that occurence of $x$ rather than before it.

\subsubsubsubsection{Actions}
An LR(0) item of the form
\begin{itemize}
    \item $N \rightarrow \alpha \bullet a \beta$ where $a$ is a terminal symbol, indicates 
    the state containing the item has a shift  \textbf{parsing} action
    \item $S' \rightarrow S \bullet$ where $S'$ is the (introduced) start symbol for the 
    grammer, inidcates the state containing the item has an  \textbf{accept} action
    \item $N \rightarrow \alpha \bullet$ where $N$ is not the (introduced) start symbol for 
    the grammar, indicates the state containing the item has a parsing action 
    \textbf{reduce $N \rightarrow \alpha$}
\end{itemize}
A shift action at end-of-file is an error, as is an accept action when the input is not at 
end-of-file.

\subsubsubsubsection{Example}
\begin{equation*}
    \begin{aligned}[c]
        S \rightarrow A
    \end{aligned}
    \qquad 
    \begin{aligned}[c]
        A \rightarrow (\ A\ )
    \end{aligned}
    \qquad 
    \begin{aligned}[c]
        A \rightarrow a
    \end{aligned}
\end{equation*}
\image{LR(0) Automaton Example.png}
\begin{center}
    \begin{tabular}{|lr|l|}
        \hline
        Parsing stack \hspace{0.5cm} & Input & Parsing action \\
        \hline
        \$0 & ((a))\$ & shift \\
        \$0(2 & (a))\$ & shift \\
        \$0(2(2 & a))\$ & shift \\
        \$0(2(2a3 & ))\$ & reduce $A \rightarrow a$ \\
        \$0(2(2A4 & ))\$ & shift \\
        \$0(2(2A4)5 & )\$ & reduce \( A \rightarrow (A)\) \\
        \$0(2A4 & )\$ & shift \\
        \$0(2A4)5 & \$ & reduce \( A \rightarrow (A)\) \\
        \$0A1 & \$ & accept \\
        \hline
    \end{tabular}
\end{center}


\subsubsubsubsection{Parsing Conflicts}
If a state in an LR(0) parsing automaton has more than one action, 
there is a parsing action conflict.

A grammar is LR(0) if none of the states in its LR(0) parsing 
automaton contains a parsing action conflict.

\subsubsubsection{LR(1) Grammars}
An LR(1) parsing item is a pair
\begin{align*}
    [N \rightarrow \alpha \bullet \beta, T]
\end{align*}
consisting of
\begin{itemize}
    \item an LR(0) parsing item $N \rightarrow a \bullet \beta$, and
    \item a set $T$ of terminal symbols called a look-ahead set.
\end{itemize}
The above item indicates that in matchin $N$, $\alpha$ has been matched and $\beta$ is yet 
to be matched, in a context in which $N$ can be followed by a terminal symbol in the set $T$.

\subsubsubsubsection{Parsing Automaton}
The kernal item of the initial state is
\begin{align*}
    [S' \rightarrow \bullet S, \$]
\end{align*}
where $S$ is the start symbol of the grammar and we introduce a fresh replacement start 
symbol $S'$ and production $S' \rightarrow S$. This fresh production is used to determine 
when parsing has completed.

\subsubsubsubsection{Derived Items}
If a state has an LR(1) item of the form
\begin{align*}
    [N \rightarrow \alpha \bullet M \beta, T]
\end{align*}
where the nonterminal M has productions
\begin{align*}
    M \rightarrow \alpha_1 | \alpha_2 | \dots | \alpha_m
\end{align*}
and $T = \{a_1, a_2, \dots, a_n\}$, then the state also includes the derived items
\begin{align*}
    [N \rightarrow \bullet \alpha_1, T]
    \dots
    [N \rightarrow \bullet \alpha_m, T]
\end{align*}
where if $\beta$ is not nullable $T' = First(\beta)$ and if $\beta$ is nullable, 
$T' = First(\beta) - \{\epsilon\} \cup T$.

\subsubsubsubsection{Parsing Actions}
An LR(1) item of the form
\begin{itemize}
    \item $[N \rightarrow \alpha \bullet a \beta, T]$, where $a$ is a terminal symbol, 
    indicates the state containing the item has a \textbf{shift} parsion action if the
    next input $x$ is $a$
    \item $[S' \rightarrow \S \bullet, \$]$, where $S'$ is the added start symbol for 
    the grammar, indicates the state containing the item has an \textbf{accept} action 
    if there is no more input.
    \item $[N \rightarrow \alpha \bullet, T]$, where $N$ is not $S'$, indicates the state 
    containing the item has a parsing action \textbf{reduce $N \rightarrow \alpha$} if 
    the next input $x$ is in $T$.
\end{itemize}

\subsubsubsubsection{Example}
\begin{equation*}
    \begin{aligned}[c]
        S \rightarrow E
    \end{aligned}
    \qquad 
    \begin{aligned}[c]
        E \rightarrow E + n
    \end{aligned}
    \qquad 
    \begin{aligned}[c]
        E \rightarrow n
    \end{aligned}
\end{equation*}
\image{LR(1) Automaton Example.png}

\subsubsubsubsection{Parsing Action Conflicts}
If a state in an LR(1) parsing automaton has more than one action for a look-ahead 
terminal symbol, there is a parsing action conflict.

A grammar is LR(1) if none of the states in its LR(1) parsing automaton contains a parsing
action conflict.


\subsubsubsection{LALR(1) Parsing}
\begin{itemize}
    \item An LALR(1) parsing automaton can be formed from an LR(1) parsing automaton by 
    merging states that have identical sets of LR(0) items but possibly different look-ahead 
    sets.
    \item A grammar is LALR(1) if none of the states in its LALR(1) parsing automaton contains 
    a parsing action conflict.
\end{itemize}