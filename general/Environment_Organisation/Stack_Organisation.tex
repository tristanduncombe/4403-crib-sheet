\subsubsection{Stack Organisation}
\subsubsubsection{Definition}
A stack machine consists of the following chunks of memory.
\begin{itemize}
    \item \textbf{stack} - the portion of memory used for both calculating the values of 
    expressions as well as storing activation records for every active procedure call.
    \item \textbf{heap} - the portion of memory for dynamic allocation of Objects.
    \item \textbf{code space} - the portion of memory where the machine instructions are 
    stored.
\end{itemize}
The stack and heap are stored in one contiguous area of memory. The stack grows from the 
bottom (address 0) and the heap grows down from the top.

The machine has four special registers:
\begin{itemize}
    \item \textbf{stack pointer} - contains the address for the top of the stack + 1
    \item \textbf{stack limit} - contains the address of the upper limit for the stack 
    and the bottom of the heap
    \item \textbf{frame pointer} - contains the address of teh stack frame for the 
    current procedure
    \item \textbf{program counter} - contains the address of the next machine instruction
\end{itemize}

\subsubsubsection{Procedures}
Each time a procedure is entered via a call, the machine must keep track of information 
such as the return address and the local variables for that call in a procedure frame. 
This frame is stored on the stack.

A pointer to the current procedures' frame is stored in a globally known location.
\subsubsubsubsection{Calling a Procedure}
\begin{itemize}
    \item parameters to the procedure are pushed to the stack
    \item a static link is pushed onto the stack
    \item the current frame pointer is pushed onto the stack to create the dynamic link
    \item the frame pointer is set so that it contains the address of the start of the 
    new stack frame
    \item the current value of the program counter is pushed onto the stack to form the 
    return address
    \item the program counter is set to the address of the procedure
    \item space is allocated on the stack for any local variables
\end{itemize}
\begin{center}
    \image[\columnwidth/2]{Stack Frame.png}
\end{center}

\subsubsubsubsection{Returning From a Procedure}
\begin{itemize}
    \item the program counter is set to the return address in the current activation record
    \item the frame pointer is set to the dynamic link
    \item the stack pointer is set so that all the space used by the stack frame (but not 
    parameters) is popped from the stack
    \item execution continues at the instruction addressed by the (restored) program counter
    \item after return, the calling procedure handles deallocating any parameters
\end{itemize}

\subsubsubsubsection{Local Variables}
Local variables are stored within the procedures frame. They are accessed by an offset 
relative to the frame pointer.

\subsubsubsubsection{Non-local Variables}
To allow access to variables outside of the enclosing procedure, the stack frame for a 
procedure includes a \textit{static link} which contains the address of the stack frame 
for the enclosing procedure (the procedure in which the enclosed procedure is defined).

To access the non-local variable $n$ from a procedure, we continuesly access the static link 
of the enclosing procedures until we are in the procedure in which $n$ is defined. We can then
add the offset to the variable $n$ to get the final address of the non-local variable.

\subsubsubsection{Parameters}
\begin{itemize}
    \item \textbf{Formal parameters} are the paremeters used in the declaration of the procedure 
    in its header.
    \item \textbf{Actual parameters} are the actual parameters passed to a procedure on a call.
\end{itemize}
\subsubsubsubsection{Call-by-value}
\begin{itemize}
    \item parameters are expressions that are evaluated and coerced to the type of the formal 
    paramter.
    \item the values of the parameters are loaded onto the stack as part of the calling sequence.
    \item Once the procedure is called and the stack frame has been established, the formal 
    parameters of the procedure can be accessed like local variables
    \item Accesses to the formal parameter access the location on the stack containing the value of the
    corresponding actual parameter.
\end{itemize}

\subsubsubsubsection{Call-by-const}
The formal parameter acts as a read-only local variable that is assigned the value of the actual 
parameter expression.

\subsubsubsubsection{Call-by-result}
The formal parameter acts as a local variable whose final value is assigned to the actual parameter 
variable.

\subsubsubsubsection{Call-by-value-result}
A single parameter acts as both a value and a result parameter.

\subsubsubsubsection{Call-by-reference}
The formal parameter is the address of the actual parameter variable. All references to the formal 
parameter are applied to the actual parameter variable immediately.

\subsubsubsubsection{Call-by-sharing}
The same as call-by-value, but what is passed is a reference to an object (e.g. Java) rather than 
the values of the object.

\subsubsubsubsection{Call-by-name}
The actual parameter expression is evaluated every time the formal parameter is accessed.

\subsubsubsubsection{Passing Procedures as Parameters}
The address of the procedure as well as the static link for the procedure's environment is passed.

\subsubsubsection{Function Results}
\begin{itemize}
    \item A function can return a result that can be used as part of an expression.
    \item The result of a function call should be left on top of the stack after the stack frame 
    is removed.
    \item To ensure the returned value is before the stack frame, free space is allocataed for 
    the result before the parameters are loaded onto the stack and the frame is set up.
\end{itemize}

\subsubsubsubsection{Returning Procedures}
Return the address of the procedure as well as the static link for the procedures environment.
This requires the environment of the returned procedure to be maintained which makes the simple
stack-based allocation of frames insufficient.

\subsubsubsection{Variable Aliasing}
\subsubsubsubsection{Parameter Aliasing}
In languages with call-by-reference, the same variable can be passed to two (or more) different
parameters leading to variable aliasing.

\subsubsubsubsection{Global Variable Aliasing}
If a variable is passed as a reference parameter to a procedure that can access the same variable
as a global variabel, then within the procedure there are two aliases for the same variable.

\subsubsubsection{Pointer Aliasing}
\subsubsubsubsection{Parameter Aliasing}
In languages with call by sharing, the same reference to an object can be passed to two different 
parameters leading to one having two aliases for the same reference.

\subsubsubsubsection{Global Variable Aliasing}
If a reference that is passed as a parameter to a procedure is also directly accessible as a 
global variable from the procedure. this leads to the procedure having two aliases for the one 
reference.