\subsubsection{Stack Organisation}
\subsubsubsection{Definitions}
\begin{itemize}
    \item \textbf{stack pointer} - contains the address for the top of the stack + 1
    \item \textbf{stack limit} - contains the address of the upper limit for the stack 
    \& the bottom of the heap
    \item \textbf{frame pointer} - contains the address of teh stack frame for the 
    current procedure
    \item \textbf{program counter} - contains the address of the next machine instruction
\end{itemize}

\subsubsubsection{Procedures}
A pointer to the current procedures' frame is stored in a globally known location.
\subsubsubsubsection{Calling a Procedure}
{
    \setlength{\columnseprule}{0pt}
    \begin{multicols}{2}
    \begin{itemize}
        \item parameters to the procedure are pushed to the stack
        \item a static link is pushed onto the stack
        \item the current frame pointer is pushed onto the stack to create the dynamic link
        \item the frame pointer is set so that it contains the address of the start of the 
        new stack frame
    \end{itemize}
        \columnbreak
    \begin{center}
        \image[\columnwidth]{Stack Frame.png}
    \end{center}
    \end{multicols}
    \begin{itemize}
        \item the current value of the program counter is pushed onto the stack to form the 
        return address
        \item the program counter is set to the address of the procedure
        \item space is allocated on the stack for any local variables
    \end{itemize}
}

\subsubsubsubsection{Returning From a Procedure}
\begin{itemize}
    \item the program counter is set to the return address in the current activation record
    \item the frame pointer is set to the dynamic link
    \item the stack pointer is set so that all the space used by the stack frame (but not 
    parameters) is popped from the stack
    \item execution continues at the instruction addressed by the (restored) program counter
    \item after return, the calling procedure h\&les deallocating any parameters
\end{itemize}

\subsubsubsubsection{Local Variables}
Local variables are stored within the procedures frame. They are accessed by an offset 
relative to the frame pointer.

\subsubsubsubsection{Non-local Variables}
We continuesly access the static link of the enclosing procedures until we are in the procedure 
in which $n$ is defined. We can then add the offset to the variable $n$ to get the final address 
of the non-local variable.

\subsubsubsection{Parameters}
\begin{itemize}
    \item \textbf{Formal parameters} are the paremeters used in the declaration of the procedure 
    in its header.
    \item \textbf{Actual parameters} are the actual parameters passed to a procedure on a call.
\end{itemize}
\textbf{Call-by-value} ---
The formal parameters are variables which are assigned a copy of the value of the actual parameter.

\textbf{Call-by-const} ---
The formal parameter acts as a read-only local variable that is assigned the value of the actual 
parameter expression.

\textbf{Call-by-result} ---
The formal parameter acts as a local variable whose final value is assigned to the actual parameter 
variable.

\textbf{Call-by-value-result} ---
A single parameter acts as both a value \& a result parameter.

\textbf{Call-by-reference} ---
The formal parameter is the address of the actual parameter variable. All references to the formal 
parameter are applied to the actual parameter variable immediately.

\textbf{Call-by-sharing} ---
The same as call-by-value, but what is passed is a reference to an object (e.g. Java) rather than 
the values of the object.

\textbf{Call-by-name} ---
The actual parameter expression is evaluated every time the formal parameter is accessed.

\textbf{Passing Procedures as Parameters} ---
The address of the procedure as well as the static link for the procedure's environment is passed.

\subsubsubsection{Function Results}
To ensure the returned value is before the stack frame, free space is allocataed for 
the result before the parameters are loaded onto the stack \& the frame is set up.

\textbf{Returning Procedures} ---
Return the address of the procedure as well as the static link for the procedures environment.
This requires the environment of the returned procedure to be maintained which makes the simple
stack-based allocation of frames insufficient.

\subsubsubsection{Variable Aliasing}
\textbf{Parameter Aliasing} ---
In languages with call-by-reference, the same variable can be passed to two (or more) different
parameters leading to variable aliasing.

\textbf{Global Variable Aliasing} ---
If a variable is passed as a reference parameter to a procedure that can access the same variable
as a global variabel, then within the procedure there are two aliases for the same variable.

\subsubsubsection{Pointer Aliasing}
\textbf{Parameter Aliasing} ---
In languages with call by sharing, the same reference to an object can be passed to two different 
parameters leading to one having two aliases for the same reference.

\textbf{Global Variable Aliasing} ---
If a reference that is passed as a parameter to a procedure is also directly accessible as a 
global variable from the procedure. this leads to the procedure having two aliases for the one 
reference.