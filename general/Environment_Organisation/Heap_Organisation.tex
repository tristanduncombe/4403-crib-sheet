\subsubsection{Heap Organisation}
\subsubsubsubsection{Issues w/ Explicit Deallocation}
\begin{itemize}
    \item \textbf{Dangling references} --- an object can be freed via one reference but 
    still be accessible via other references.
    \item \textbf{Memory leaks} --- all references to an object are removed but the object 
    was not freed.
    \item \textbf{Memory fragmentation} --- the memory consists of alternating alloctated 
    \& free areas, with no large area of free memory.
    \item \textbf{Locality of reference} --- how spread out the memory is in the address space.
\end{itemize}

\subsubsubsection{Garbage Collection}
An object is accessible if it can be reached either
\begin{itemize}
    \item directly via a reference in a global or local variable, or
    \item indirectly via a reference in an object that is accessible.
\end{itemize}
Garbage collection determined which objects are not accessible \& recovers the space used 
by them.

\subsubsubsubsection{Mark-\&-sweep}
\begin{itemize}
    \item a phase that \textbf{marks} all the accessible objects
    \item a phase that \textbf{sweeps} up the objects left unmarked \& adds them to the 
    free list
    \item quick since it only needs to mark memory as free (no moving required)
    \item results in fragmentation since it doesn't compact memory
    \item can use a lot of memory if heavily fragmented with small free areas
\end{itemize}

\subsubsubsubsection{Stop-\&-copy}
\begin{itemize}
    \item divides the available memory into two spaces
    \item memory is allocated sequentially from one space until it runs out
    \item garbage collection consists of relocating all accessible objects from the first 
    space to the second space
    \item slow since it must copy all words from one heap to the other
    \item operation compacts memory so low fragmentation
    \item uses double the available memory since 2 identically sized heaps are needed
\end{itemize}
Overcomes memory fragmentation problems, but copying can be expensive in time.

\subsubsubsubsection{Generational Schemes}
Generational schemes use a scheme similar to the stop-\&-copy scheme, but make us of more 
spaces
\begin{itemize}
    \item the spaces are organised based on the length of time its objects have survived
    \item older objects are migrated to an old object space \& newer objects go in the a 
    new object space
    \item the newer the space, the more frequently it is garabage collected
    \item more efficient since avoids repeatedly copying old objects
    \item memory fragmentation low since objects are compacted when moved between generations
    \item uses high-ish memory utilisation since multiple heaps are needed (older heaps can use less memory)
\end{itemize}

\subsubsubsubsection{Issues w/ Garbage Collection}
\begin{itemize}
    \item Garbage collection has a greater time overhead than explicit deallocation
    \item Response time can vary because garbage collection can happen at any time 
    \& is time expensive
    \item Real-time response hard to guarantee
\end{itemize}

\subsubsubsubsection{On-the-fly}
\begin{itemize}
    \item Garbage collection process takes place incrementally, interleaed with 
    execution of the program.
    \item Each time an object is allocated, the garbage collector can do a bit 
    of work.
\end{itemize}

\subsubsubsubsection{Concurrent}
\begin{itemize}
    \item The garbage collector runs concurrently with the program on a separate 
    processor.
\end{itemize}