\subsubsection{Heap Organisation}
\subsubsubsection{Dynamic Memory Allocation}
Dynamic allocation of objects takes place on the $heap$, an area of memory seperate 
from the stack.

\subsubsubsection{Dynamic Memory Deallocation}
There are two main forms of memory deallocation used in programming languages:
\begin{itemize}
    \item \textbf{Explicit deallocation} in which there is a function to explicitly 
    deallocate and object
    \item \textbf{Garbage collection} in which the space used by unreachable objects 
    is reclaimed by a garbage collector.
\end{itemize}

\subsubsubsubsection{Explicit Memory Deallocation}
\begin{itemize}
    \item freed memory is returned to the "free list"
    \item if the memory freed is adjacent to other blocks of memory in the free list, 
    the adjacent free blocks are merged into a single larger free block to reduce 
    fragmentation
    \item the freed memory is available for re-allocation
\end{itemize}

\subsubsubsubsection{Issues w/ Explicit Deallocation}
\begin{itemize}
    \item \textbf{Dangling references} - an object can be freed via one reference but 
    still be accessible via other references
    \item \textbf{Memory leaks} - all references to an object are removed but the object 
    was not freed.
    \item \textbf{Memory fragmentation} - the memory consists of alternating alloctated 
    and free areas, with no large area of free memory.
    \item \textbf{Locality of reference} - how spread out the memory is in the address space.
\end{itemize}
All of these issues can be solved with a garbage collector.

\subsubsubsection{Garbage Collection}
An object is accessible if it can be reached either
\begin{itemize}
    \item directly via a reference in a global or local variable, or
    \item indirectly via a reference in an object that is accessible.
\end{itemize}
Garbage collection determined which objects are not accessible and recovers the space used 
by them.

\subsubsubsubsection{Mark-and-sweep}
Mark and sweep consists of
\begin{itemize}
    \item a phase that \textbf{marks} all the accessible objects
    \item a phase that \textbf{sweeps} up the objects left unmarked and adds them to the 
    free list
\end{itemize}

\subsubsubsubsection{Stop-and-copy}
\begin{itemize}
    \item divides the available memory into two spaces
    \item memory is allocated sequentially from one space until it runs out
    \item garbage collection consists of relacating all accessible objects from the first 
    space to the second space
    \item because the objects are allocated sequentially in the second space, they are 
    compacted
    \item when the copy is completed, the roles of the spaces are swapped for the next 
    garbage collection
\end{itemize}
Overcomes memory fragmentation problems, but copying can be expensive in time.

\subsubsubsubsection{Generational Schemes}
Generational schemes use a scheme similar to the stop-and-copy scheme, but make us of more 
spaces
\begin{itemize}
    \item the spaces are organised based on the length of time its objects have survived
    \item the ages of an object is the number of times it has been collected
    \item older objects are migrated to an old object space
    \item newer objects go in the new object space
    \item the objects in the old object space is garbage collected
    \item the old object space may have to be garbage colelcted at some stage
\end{itemize}

\subsubsubsubsection{Issues w/ Garbage Collection}
\begin{itemize}
    \item Garbage collection has a greater time overhead than explicit deallocation
    \item Response time can vary because
        \item garbage collection can happen at any time
        \item it can take a significant amount of time
    \item Real-time response hard to guarantee
        \item Pre-allocating all objects necessary
        \item Use incremental on-the-fly garbage collected
        \item Use concurrent garbage collector
    \item It is complex and tricky to get correct 
\end{itemize}

\subsubsubsubsection{On-the-fly}
\begin{itemize}
    \item Garbage collection process takes place incrementally, interleaed with 
    execution of the program.
    \item Each time an object is allocated, the garbage collector can do a bit 
    of work.
    \item Avoids the program stopping for a longish time interval to garbage 
    collect in one go, which can be a problem when real-time response is required.
\end{itemize}

\subsubsubsubsection{Concurrent}
\begin{itemize}
    \item The garbage collector runs concurrently with the program on a separate 
    processor.
\end{itemize}